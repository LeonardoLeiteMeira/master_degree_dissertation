\chapter{Introduction}\label{cap:intro}
% To check how acronyms work, just try to write \gls{ESTiG}.

% \begin{figure}[htbp]
%     \begin{center}
%     \includegraphics[scale=0.05]{images/imagem01}
%     \end{center}
%     \caption{Example of figure.}
%     \label{fig:graficosubscricoesmoveis}
% \end{figure}


\section[Contextualização do Problemas]{Contextualização do Problemas}

%TODO artigo que mostra o impacto da tecnologia na industria, forçando a evolução do ambiente industrial

%TODO artigo que mostra a diferença entre empresas que monitoram processos e empresas que nao fazem esse monitoramento 
No cenário industrial contemporâneo, a busca constante por eficiência e inovação tornou-se um pilar essencial para a competitividade e sustentabilidade financeira das empresas. À medida que a tecnologia avança a produção industrial sofre uma evolução para se manter competitiva perante ao mercado. Nesse contexto, o monitoramento e a otimização das máquinas em linhas de produção tornam-se cruciais para garantir um funcionamento eficaz e prevenir potenciais paralisações ou falhas operacionais.

%TODO artigo que mostra sobre a quantidade de empresas que ainda usam pouca tecnologia no ambiente produtivo
No entanto, a tradição das práticas industriais muitas vezes é caracterizada por inspeções manuais e sistemas de monitoramento desatualizados, que não conseguem fornecer informações em tempo real ou análises aprofundadas sobre o desempenho das máquinas. Esse descompasso tecnológico pode resultar em perdas significativas, em termos de produção, de recursos financeiros e manutenção de máquinas.

%TODO artigo sobre a quantidade massiva de dados que é produzida em excesso
Além disso, com a crescente integração de sistemas de IoT e a proliferação de sensores avançados, existe uma quantidade imensa de dados sendo gerada continuamente. No entanto, sem a infraestrutura adequada para coletar, armazenar e analisar esses dados, as empresas podem se encontrar sobrecarregadas, incapazes de extrair insights significativos que poderiam informar decisões estratégicas e operacionais.

Portanto, a necessidade identificada é desenvolver um sistema robusto que possa receber dados de sensores que coletam dados das maquinas em tempo real, armazená-los de maneira eficiente em um data lake e apresentá-los através de um dashboard, transformando a maneira como as empresas monitoram e otimizam suas linhas de produção, garantindo não apenas eficiência, mas também uma abordagem proativa à manutenção e gestão industrial. Esse sistema não apenas forneceria informações em tempo real sobre o status e o desempenho das máquinas, mas também permitiria análises históricas, ajudando gestores e técnicos a identificar tendências e falhas, além de otimizar a produção, minimizando perdas na produção e otimizando os ganhos financeiros.
