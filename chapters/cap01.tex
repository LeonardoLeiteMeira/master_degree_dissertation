\chapter{Introduction}\label{cap:intro}

\section{Framework}

In the contemporary industrial scenario, the constant pursuit of efficiency and innovation has become an essential pillar for the competitiveness and financial sustainability of companies \cite{Bonilla2018}. As technology advances, companies face pressure to remain competitive in the market \cite{ashraf2020optimization}. In this context, monitoring and optimizing machines in production lines become crucial to ensure effective operation and prevent potential downtime or operational failures.

However, the tradition of industrial practices is often characterized by manual inspections and outdated monitoring systems that fail to provide real-time information or in-depth analysis of machine performance. This technological gap can result in significant losses in terms of production, financial resources, and machine maintenance, considering that investment in technology and monitoring can make companies more financially efficient \cite{mabad2021rfid}.

In addition, with the increasing integration of IoT systems and the proliferation of advanced sensors, there is an immense amount of data being continuously generated, demanding more efficient processing \cite{bajaj2021iot}. However, without the proper infrastructure to store and analyze this data, companies may find themselves overwhelmed, unable to extract meaningful insights that could inform strategic and operational decisions.

In this market context, a stamping industry company sought to build projects that enable the sensorization of their machines, storage and processing of data, and visualization of this information, in order to become more competitive, efficient, and profitable. These projects were grouped within the "ATTRACT - Digital Innovation Hub for Artificial Intelligence and High-Performance Computing - Project: 101083770 — ATTRACT — DIGITAL-2021-EDIH-01" context, with funding from the "Digital Europe Programme (DIGITAL) - DIGITAL-2021-EDIH-INITIAL-01 — Initial Network of European Digital Innovation Hubs", referred to as the \texttt{Attract Project} in this document.

\section{Objectives}
Given the previous framework, the identified need is to develop a robust system that can receive data from sensors that collect real-time data from machines, store them efficiently in a data lake, and present them through a dashboard, transforming the way the company monitors and optimizes its production line, ensuring not only efficiency but also a proactive approach to maintenance and industrial management. This system would not only provide real-time information about the status and performance of the machines, but it would also allow for historical analyses, helping managers and technicians to identify trends and failures, as well as optimize production, minimizing production losses and maximizing financial gains.

\section{Document Structure}
This master's dissertation is organized starting with the Introduction, where the problem and scope are presented. This section contextualizes the need for industrial modernization, highlighting the importance of machine monitoring and optimization in the industry. The relevance of the study is then justified, focusing on the increasing demand for data analysis in the market.

This is followed by the Literature Review, which presents an analysis of works and concepts related to the project. This section encompasses similar works already carried out in the parts of real-time data reception, processing, and alerts.

The Methodology discusses the choice of specific technologies used in the project, detailing the data storage method, and the process adopted for the software development, and also describes the strategies for managing the project activities.

In the chapter on System Architecture, the technical specifics of the proposed system are discussed. From the general system diagram to the way each part of the system is organized, making clear how each part is organized and how the interaction between them occurs.

In the Implementation chapter, the technical aspects of the system are delved into. Here, each component of the database, the \gls{API}, the data processing module, the data receiving module, and the frontend, are detailed. This chapter aims to show in a practical way how the architecture detailed in the previous chapter was implemented.

In regard to the System Characteristics from a functional point of view, the dissertation has a section that focuses on demonstrating the practical application of the software, showing how each feature works and how user interactions occur.

In Results and Evaluation, the results obtained during the project are presented, highlighting the main achievements and benefits observed in the implementation of the proposed system. The Conclusion and Future Works section summarizes the key points of the project, identifies the limitations of the current system, and suggests possible directions for continuity and future implementations.
