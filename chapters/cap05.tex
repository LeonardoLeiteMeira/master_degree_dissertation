\chapter{Implementação}\label{cap:implementation}

%TODO - Itens de implementação para adicionar;
% - Classe Singleton -
% - Classe ThreadManager -
% - Classe Datatype -
% - MetadataRepository -
% - iot_collection_parser.py - 
% - Pydantic - 
% - Envio de notificações via web socket - 


Após o capítulo onde a arquitetura do software foi detalhada, este capítulo é focado em explicar como tal arquitetura foi implementada, pois enquanto o primeiro descreve a estrutura e a organização, este foca nas ações técnicas adotadas para fazer essa estrutura funcionar.

Para uma análise mais estruturada e detalhada, este capítulo foi dividido em seções específicas para cada componente do sistema. São elas:

\begin{itemize}
    \item \textbf{Implementação do banco de dados}: Esta seção abordará os detalhes técnicos do design do banco de dados, esquemas adotados e como as informações são armazenadas e recuperadas.
    
    \item \textbf{Implementação do módulo de recebimento de dados}: Esta seção detalhará como os dados são recebidos, validados e processados antes de serem armazenados e disponibilizados para os usuários.
    
    \item \textbf{Implementação da API}: Aqui, a estrutura da API será discutida, passando pelos endpoints fornecidos, a lógica por trás de cada um e as camadas utilizadas.
    
    \item \textbf{Implementação do módulo de processamento de dados}: É abordado o tratamento dos dados recebidos pelos sensores, e como é feito a analise estatística que gera as informações apresentadas nos gráficos.
    
    \item \textbf{Implementação do frontend}: Por fim, a interface com o usuário será discutida, explicando como os dados são estruturados apresentados e apresentados em tela.
\end{itemize}


\section[Implementação do banco de dados]{Implementação do banco de dados}


Dentro da implementação do sistema, o MongoDB foi usado para armazenar todas as informações do sistema. Este banco de dados, orientado a documentos, permitiu uma organização flexível dos dados, facilitando o armazenamento de diferentes dados que podem ser recebidos pelo modulo de recebimento de dados, e facilitando a criação de camadas de processamento. A estruturação dos bancos de dados e suas respectivas coleções foi pensada para facilitar tanto a inserção quanto a consulta de informações.

Em relação à organização dos dados, os seguintes bancos de dados foram criados:

\begin{itemize}
    \item \textbf{Users}: Armazena informações referentes aos usuários. Possui coleções que registram tentativas de login, detalhes pessoais dos usuários e tokens associados a eles.
    
    \item \textbf{Notification}: Destinado às notificações do sistema. Atualmente, este banco contém apenas notificações associadas aos alertas das máquinas, gerados pelos dados recebidos dos sensores junto com os parâmetros armazenados.
    
    \item \textbf{Downtime}: Armazena duas coleções, uma com os dados lidos das planilhas de parada das máquinas, e outro com esses dados tratados. Esse banco de dados com essas coleções são apenas para simular como ficaria os dados de parada das maquinas, caso eles fosses inseridos no sistema.
    
    \item \textbf{Raw Data}: Este banco é dedicado ao armazenamento de dados brutos oriundos de diferentes sensores. Cada tipo de sensor, como os sensores de pressão, tem sua própria coleção, garantindo um agrupamento das informações que facilita a análise.
    
    \item \textbf{Processed Data}: Como o próprio nome sugere, armazena dados que já passaram por uma etapa de processamento. Assim, dados interpretados de diferentes sensores são separados em coleções específicas, como os de pressão em uma e os de voltagem em outra.
    
    \item \textbf{Metadados}: Dedicado à armazenagem de metadados do sistema. Até o momento, a única coleção presente é a "AlertParameter", que reúne parâmetros utilizados para gerar alertas associados a cada sensor.
\end{itemize}

Com esta estruturação, busca-se não apenas organizar de forma lógica os dados, mas também otimizar operações de consulta e garantir uma expansão simplificada à medida que novas necessidades de armazenamento emergem no sistema. 

A implementação do acesso ao banco de dados está detalhada na seção de implementação da API, em ~\ref{subsubsec:DatabaseImpl}.



\section{Implementation of the data receiving module}\label{sec:Implementation of the data receiving module}

In the process of implementing the system, one of the essential steps was the development of a module intended for receiving data from IoT sensors. This reception is carried out via a multicast connection, an efficient approach to handle the transmission of messages to multiple recipients simultaneously.

This module is responsible for establishing the multicast connection to receive the data, performing the conversion of the received data according to the predefined protocol, making the data available to be displayed in real time for connected users, checking if it generates any type of alert (and if it does, notifying users about it by creating a notification), and saving the generated information in the database.

\subsection[Connection and data reception]{Connection and data reception}\label{subsec:Connection and data reception}

The \texttt{SensorConnection} class has the main responsibility of creating a socket, staying connected to receive messages and interpreting them. The structure and operation of this class are detailed below.

The \texttt{SensorConnection} class is initiated with the creation of an IPv4 and UDP socket:

\begin{Verbatim}[fontsize=\small, baselinestretch=0.8]
class SensorConnection:
    def __init__(self):
        self.sock = socket.socket(socket.AF_INET, socket.SOCK_DGRAM)
\end{Verbatim}

To ensure that the system is constantly listening to multicast messages from the sensors, the \texttt{listen\_multicast\_messages} method was defined within this class. It starts with the creation of the connection and initiates the message reading process, also managing possible disconnections and reestablishing the connection when necessary:

\begin{Verbatim}[fontsize=\small, baselinestretch=0.8]
    async def listen_multicast_messages(self, save_data_func):
        self.__create_connection()
        while True:
            await self.__start_read_messages(save_data_func)
            self.sock.close()
            time.sleep(1)
            self.__reconnect()
\end{Verbatim}

The function \texttt{\_\_create\_connection} is responsible for establishing and configuring the initial connection with the multicast group, and within the infinite loop, the reception of messages is initiated with the \texttt{\_\_start\_read\_messages} method. When this method is finished, the socket connection is closed, and then reconnected to resume reading the messages. The call to the \texttt{time.sleep(1)} function is used to have a small interval between one call and another so as not to make a very large number of calls in case there is some kind of problem.

Below, each of the functions called within this method is detailed.

\subsubsection[Create connection method]{Create connection method}

\begin{Verbatim}[fontsize=\small, baselinestretch=0.8]
def __create_connection(self):
    self.sock.setsockopt(socket.SOL_SOCKET, socket.SO_REUSEADDR, 1)

    server_address = ('', SENSOR_MULTICAST_PORT)
    self.sock.bind(server_address)

    multicast_group = SENSOR_MULTICAST
    group = socket.inet_aton(multicast_group)
    mreq = struct.pack('4sL', group, socket.INADDR_ANY)
    self.sock.setsockopt(socket.IPPROTO_IP,
        socket.IP_ADD_MEMBERSHIP,
        mreq)
\end{Verbatim}

Initially, the socket is configured to allow multiple connections on a single address. The \texttt{SO\_REUSEADDR} option is set to the value 1, allowing more than one socket to bind to the same address, which is especially useful in multicast connection contexts:

\begin{Verbatim}[fontsize=\small, baselinestretch=0.8]
    self.sock.setsockopt(socket.SOL_SOCKET, socket.SO_REUSEADDR, 1)
\end{Verbatim}

After this, the socket is bound to a specific multicast address and port. It is important to note that the first argument in the server address definition is left blank. This approach ensures that the system is connecting to all available network interfaces, providing broad connection coverage:

\begin{Verbatim}[fontsize=\small, baselinestretch=0.8]
    server_address = ('', SENSOR_MULTICAST_PORT)
    self.sock.bind(server_address)
\end{Verbatim}

Finally, to effectively join the multicast group, some steps are performed. The multicast IP address is first converted to binary format with the call to \texttt{socket.inet\_aton}. Then, this address and the local address (represented by \texttt{socket.INADDR\_ANY}) are packed into a data structure by \texttt{struct.pack}. This structure is used to specify to the socket that it should join a multicast group in \texttt{self.sock.setsockopt}. The \texttt{IP\_ADD\_MEMBERSHIP} option is set and the previously created structure is passed as an argument, concluding the connection to the multicast group:

\begin{Verbatim}[fontsize=\small, baselinestretch=0.8]
    multicast_group = SENSOR_MULTICAST
    group = socket.inet_aton(multicast_group)
    mreq = struct.pack('4sL', group, socket.INADDR_ANY)
    self.sock.setsockopt(socket.IPPROTO_IP, socket.IP_ADD_MEMBERSHIP, mreq)
\end{Verbatim}

These operations ensure that the socket is configured and connected to the multicast group, ready to receive messages from multiple sources simultaneously.


\subsubsection[Reconnect Method]{Reconnect Method}
\begin{Verbatim}[fontsize=\small, baselinestretch=0.8]
def __reconnect(self):
    self.sock = socket.socket(socket.AF_INET, socket.SOCK_DGRAM)
    self.__create_connection()
\end{Verbatim}

In situations where the connection to the sensors is interrupted, the \texttt{\_\_reconnect} method is called to try to reestablish the connection, creating a new instance of the socket and calling the \texttt{\_\_create\_connection} function again, detailed earlier.

\subsubsection[Method start read messages]{Method start read messages}

\begin{Verbatim}[fontsize=\small, baselinestretch=0.8]
async def __start_read_messages(self, save_data_func):
    while True:
        try:
            data, address = self.sock.recvfrom(1024)
            result = self.__parse_multicast_message(data)
            if not type(result) == str:
                await save_data_func(result)
        except Exception as e:
            print(f"Error: {e}")
            break
\end{Verbatim}

After the settings are made, the messages are continuously read and processed by the \texttt{\_\_start\_read\_messages} function. During this process, each message is processed by the \texttt{\_\_parse\_multicast\_message} method, and if it is in the correct format, it is passed to a function that will save and make it available for the \gls{API} to stream to connected users.

If there is any problem in the execution of this method, it is terminated and returns to the \texttt{listen\_multicast\_messages}, where the socket is closed and a new connection is established by the \texttt{\_\_reconnect} method.

\subsubsection[Parse Multicast Messages Method]{Parse Multicast Messages Method}

\begin{Verbatim}[fontsize=\small, baselinestretch=0.8]
def __parse_multicast_message(self, data):
        (machine_type_high, machine_number_low) = 
            self.__parse_bytes(data[:2])

        message_type = data[2]

        if message_type == 2:
            return "Request to publish..."

        (physical_quantity_high, sensor_number_low) = 
            self.__parse_bytes(data[3:5])
        
        (data_type_high, meaning_low) = 
            self.__parse_bytes(data[5:7])

        message_dict = {
            'Machine': {
                'Type': str(machine_type_high)+". "+MACHINE_TYPE[machine_type_high],
                'Number': machine_number_low
            },
            'Type': str(message_type)+". "+ MESSAGE_TYPE[message_type],
            'Sensor': {
                'PhysicalQuantity': PHYSICAL_QUANTITY[physical_quantity_high],
                'Number': sensor_number_low
            },
            'MeaningOfData': {
                'DataType': str(data_type_high)+". "+DATA_TYPE[data_type_high],
                'Meaning': str(meaning_low)+". "+DATA_MEANING[meaning_low]
            }
        }
        return message_dict
\end{Verbatim}

To interpret and extract information from the message received from the multicast, it is crucial to properly decode the message according to the protocol defined earlier. The implementation of this decoding is done by the \texttt{\_\_parse\_multicast\_message} method. The helper function \texttt{\_\_parse\_bytes} is used for this task, given a sequence of bytes, the function interprets the bytes using the big-endian order (where the most significant bytes come first).

\begin{Verbatim}[fontsize=\small, baselinestretch=0.8]
def __parse_bytes(self, bytes):
    data = int.from_bytes(bytes, byteorder='big')
    high_data = (data >> 8) & 0xFF
    low_data = data & 0xFF

    return (high_data,low_data)
\end{Verbatim}

Here, \texttt{data} contains the integer value of the provided bytes. The high-order byte is extracted by shifting the value 8 bits to the right and applying an "AND" operation (\&), and the low-order byte is simply obtained by applying the "AND" operation with \texttt{0xFF}.

With the ability to interpret the bytes, the main function \texttt{\_\_parse\_multicast\_message} can begin the decoding:

\begin{itemize}
    \item First, it extracts the machine type and the machine number from the first two bytes of the message.
    
    \item The third byte of the message is then interpreted as the message type. If the message type is \texttt{2}, the function will directly return a request to publish.
    
    \item Bytes 4 and 5 are interpreted as the sensor ID, which contains the physical quantity being measured and the sensor number.
    
    \item Bytes 6 and 7 are used to extract the data type and its meaning.
\end{itemize}

The extracted information is then organized into a dictionary for clear representation and easy access to the components individually:

\begin{Verbatim}[fontsize=\small, baselinestretch=0.8]
message_dict = {
    'Machine': {...},
    'Type': ...,
    'Sensor': {...},
    'MeaningOfData': {...}
}
\end{Verbatim}

This structure allows a clear and modular representation of the decoded message, making it easy to integrate and use in other parts of the system. Therefore, the return of the \texttt{\_\_parse\_multicast\_message} method is used as the result of the interpretation of the multicast message, and sent to the function received as a parameter, \texttt{save\_data\_func}.

\subsection[Verification and provision of data]{Verification and provision of data}\label{subsec:checkDataReceived}
In the data reception process, after opening the connection and the data, it is necessary to check if they are in the correct format, if it generates any alert, insert it into the database and make it available to the users connected to the system.

The \texttt{IotSensorConnection} class, which implements the \texttt{IotSensorConnectionInterface} interface, plays a key role in this module. Upon its initialization, a connection with the repository is established through the \texttt{self.\_\_repository} variable. In addition, it is responsible for the connection with the sensor, which is established through the \texttt{self.\_\_sensor\_connection}, explained earlier in section ~\ref{subsec:Connection and data reception}.

\begin{Verbatim}[fontsize=\small, baselinestretch=0.8]
class IotSensorConnection(IotSensorConnectionInterface):
    def __init__(self, respository:SensorsRepository):
        self.__repository = respository
        self.__sensor_connection = SensorConnection()
    
    def start_connection(self):
        threadManager = ThreadManager()
        threadManager.start_async_thread(self.__start_connection)
    
    async def __start_connection(self):
        await self.__sensor_connection.
            listen_multicast_messages(self.__handle_iot_data)
\end{Verbatim}

Upon initiating the connection, using the \texttt{start\_connection} method, a new thread is created through the \texttt{ThreadManager} class, explained in \ref{subsubsec:ThreadManager}. This thread invokes the \texttt{listen\_multicast\_messages} method from the \texttt{SensorConnection} class that was detailed in section ~\ref{subsec:Connection and data reception}. It is necessary to create a new thread because this module is together with the \gls{API}, and it is necessary for both processes to function at the same time, a new thread was necessary for the parallel operation of both.

\subsubsection{Data Format Verification}
To handle the received data, the \texttt{\_\_handle\_iot\_data} method is passed as an argument to \texttt{listen\_multicast\_messages} (as the save\_func argument that exists in the \texttt{SensorConnection} class).

\begin{Verbatim}[fontsize=\small, baselinestretch=0.8]
async def __handle_iot_data(self, sensor_data:dict):
    sensor_model = self.__parse_sensor_data_to_sensor_model(sensor_data)
    await self.__repository.update_current_sensor_value(
        sensor_value = sensor_model.value,
        machine = sensor_model.machine,
        date = sensor_model.date,
        sensor_type = sensor_model.type,
        sensor_number = sensor_model.sensor_number
    )

def __parse_sensor_data_to_sensor_model(self, sensor_data:dict):
    value = sensor_data["value"]
    machine = str(sensor_data["Machine"]['Number']) 
        + sensor_data["Machine"]['Type']
    date = datetime.now()
    data_type = sensor_data["Sensor"]["PhysicalQuantity"]
    sensor_number = sensor_data["Sensor"]["Number"]
    return ConnectionModelToParse(
        date=date,
        machine=machine,
        sensor_number=sensor_number,
        type=data_type,
        value=value
    )
\end{Verbatim}

This method is responsible for receiving the sensor data and converting it into a model class, named \texttt{ConnectionModelToParse}, which uses \texttt{Pydantic} to validate the information. The explanation of \texttt{Pydantic} is given in section ~\ref{subsubsec:dataModel}.

\begin{Verbatim}[fontsize=\small, baselinestretch=0.8]
class ConnectionModelToParse:
    def __init__(self,value:float,machine:str,
        date:datetime,type:Datatype,
        sensor_number:int):
            self.value = value
            self.machine = machine
            self.date = date
            self.type = type
            self.sensor_number = sensor_number
\end{Verbatim}

After this transformation, the data is forwarded to the repository. The \texttt{update\_current\_sensor\_value} method from the repository is called to check if the received data triggers any type of alert, save it in the database, update the data in memory, and perform notification checks.

\begin{Verbatim}[fontsize=\small, baselinestretch=0.8]
class SensorsRepository:
    def __init__(self):
        self.database = MongoDBIOT()
        self.iot_notification_check = IotNotificationCheck()
        self.__sensor_value = SensorValue()

    async def update_current_sensor_value(self, sensor_type:Datatype,
        sensor_value:float, machine:str, date:datetime, 
        sensor_number:int):
        alert_type = await self.__get_alert_type(sensor_value, sensor_type)
        current_value = {"machine":machine,
            "value":sensor_value, "timestamp": date,
            "alert_type":alert_type.value,
            "sensor_number":sensor_number}
        result = await self.insert_value_into_database(current_value, 
            sensor_type)
        new_id = result.inserted_id
        iot_data = IotData(
            alert_type=current_value["alert_type"],
            machine=current_value["machine"],
            timestamp=current_value["timestamp"],
            value=current_value["value"],
            id=PyObjectId(new_id),
            datatype=sensor_type,
            sensor_number=sensor_number
        )
        
        self.__sensor_value.update_sensor_value_by_type(
            iot_data,sensor_type)
        
        await self.iot_notification_check.check_iot_notification(
            iot_data)
\end{Verbatim}

\subsubsection{Alert Verification}

Within the \texttt{update\_current\_sensor\_value} method, the type of alert generated is first verified with the \texttt{\_\_get\_alert\_type} method. This method reads the parameter according to the sensor type within the system metadata, where access is explained in \ref{subsec:main}, and with it checks the alert status.

The alert status, defined by the \texttt{get\_alert\_status} function, returns as \texttt{OK} if the sensor value is less than 90\% of the value defined as a parameter, returns as \texttt{WARNING} if this value is between 90\% and 100\%, and returns as \texttt{PROBLEM} if the value returned by the sensor is greater than 100\% of the value defined as a parameter.

\begin{Verbatim}[fontsize=\small, baselinestretch=0.8]
def get_alert_status(self,sensor_value:int,
    alert_parameter:int)->AlertTypes:
        parameter = ((sensor_value/alert_parameter)*100)
        if parameter < 90:
            return AlertTypes.OK
        if parameter >= 90 and parameter < 100:
            return AlertTypes.WARNING
        if parameter >= 100:
            return AlertTypes.PROBLEM

async def __get_alert_type(self, sensor_value:float,
    sensor_type:Datatype)->AlertTypes:
        alert_parameter = await MetadataRepository()
            .get_sensor_alert_value(sensor_type)
        alert_type = self.get_alert_status(sensor_value,
            alert_parameter)
        return alert_type
\end{Verbatim}

\subsubsection{Database Registration}

With the verification of the alerts, all information has been generated, so it can now be registered in the database. The \texttt{insert\_value\_into\_database} method is used for this registration.

\begin{Verbatim}[fontsize=\small, baselinestretch=0.8]
async def insert_value_into_database(self, value:BaseIotData, type:Datatype):
    collection = sensor_name_to_raw_data_collection(type)
    return await self.database.insert_one(IOT_DATABASE,collection,value)
\end{Verbatim}

This method uses the base class of the database with already defined operations to perform the registration. Within the \texttt{update\_current\_sensor\_value} method of the repository, the return is used to keep the registered ID in memory, important for creating the \texttt{IotData} object, which is sent to the connected users, via stream, in the next step.

\begin{Verbatim}[fontsize=\small, baselinestretch=0.8]
current_value = {"machine":machine,
    "value":sensor_value,
    "timestamp": date,
    "alert_type":alert_type.value,
    "sensor_number":sensor_number}
result = await self.insert_value_into_database(current_value, sensor_type)
new_id = result.inserted_id
iot_data = IotData(...)
)
\end{Verbatim}

An important piece of information to highlight is that the name of the collection used by the \texttt{insert\_value\_into\_database} method is defined according to the established data type, using the helper function \texttt{sensor\_name\_to\_raw\_data\_collection}, explained in \ref{subsubsec:helpers}.

\subsubsection{Data update in memory}

With the alert type defined and the data registered in the database, the \texttt{SensorValue} class is used to update the information in memory. This process is done through the call \texttt{\_\_sensor\_value.update\_sensor\_value\_by\_type (iot\_data,sensor\_type)} in the \texttt{update\_current\_sensor\_value} method of the repository.

The \texttt{SensorValue} class is responsible for managing and updating the values in memory. It is noted that it uses the \texttt{Singleton} design pattern, ensuring the existence of only one instance of this class throughout the application's lifecycle, guaranteeing that there is only one instance storing the sensor information.

\begin{Verbatim}[fontsize=\small, baselinestretch=0.8]
class SensorValue(metaclass=Singleton):
    def __init__(self) -> None:
        self.machine_list:list[MachineData] = []

    def update_sensor_value_by_type(self, new_value: IotData, data_type: Datatype):
        is_new_machine = True
        for machine in self.machine_list:
            if machine.name == new_value.machine:
                is_new_machine = False
                is_new_sensor = True
                for index, sensor in enumerate(machine.sensor_data):
                    if sensor.datatype == data_type:
                        machine.sensor_data[index] = new_value
                        is_new_sensor = False
                        break

                if is_new_sensor:
                    machine.sensor_data.append(new_value)
                    break
        if is_new_machine:
            new_machine = MachineData(name=new_value.machine,sensor_data=[new_value])
            self.machine_list.append(new_machine)
\end{Verbatim}

At the time of its initialization, the \texttt{SensorValue} class initializes an empty list, \texttt{machine\_list}, which will be responsible for storing the sensor values organized by machine.

The update occurs through the \texttt{update\_sensor\_value\_by\_type} method. This method updates the sensor value in memory according to its type (\texttt{data\_type}). The update process first checks if the machine associated with the sensor already exists in the list. If so, it searches for the specific sensor within the machine's data and updates its value. If the sensor is not found, a new one is added to the list of sensors of the corresponding machine.

On the other hand, if the machine is not found in the \texttt{machine\_list}, a new instance of \texttt{MachineData} is created and added to the list, containing the machine's information and the received sensor data.

\begin{Verbatim}[fontsize=\small, baselinestretch=0.8]
class MachineData(BaseModel):
    name:str = Field(...)
    sensor_data:list[IotData] = Field([])
\end{Verbatim}

In this way, the repository sends the information to this method, and with the appropriate verification, the most updated data is kept in memory, and available to be used by the \gls{API}, enabling real-time access to sensor data.

\subsubsection{Notification Verification}
With the alert type verified, the information saved in the database, and the \texttt{IotData} object assembled, the last task of the \texttt{update\_current\_sensor\_value} method in the repository is to use the IotNotificationCheck singleton to verify the notifications regarding the operation of the machines.

The \texttt{IotNotificationCheck} class acts as an alert controller for IoT data. Upon receiving IoT data, it checks the alert status and takes appropriate measures, whether adding or removing machines or sensors from the alert list. This class is essential for monitoring and responding to real-time alert events, ensuring that associated users are notified of any abnormalities or important events detected by the IoT sensors.

Through the \texttt{check\_iot\_notification} method, the class verifies the type of alert received by the IotData object, whether the machine is in an alert state, and whether the machine's specific sensor is in an alert state. Based on this verification, the method takes one of the following actions:

\begin{enumerate}
    \item Puts a new machine in an alert state.
    \item Puts a new sensor of the machine in an alert state.
    \item Removes a sensor from the machine from the alert state. If the machine has only a single sensor in an alert state, the machine is removed from the alert state.
\end{enumerate}

\begin{Verbatim}[fontsize=\small, baselinestretch=0.8]
async def check_iot_notification(self, iot_data:IotData):
    is_alert_value = self.__is_alert_type_a_new_alert(
        iot_data.alert_type)
    machine_in_alert = self.__is_machine_in_alert_state(
        machine_name=iot_data.machine)
    machine_sensor_in_alert = self.__is_machine_sensor_in_alert_state(
        machine_in_alert,
        iot_data.datatype)
    is_machine_in_alert = machine_in_alert!=None
    if is_alert_value and is_machine_in_alert and (not machine_sensor_in_alert):
        await self.__put_new_machine_sensor_in_alert_state(
            machine_in_alert,
            iot_data.datatype)
    if is_alert_value and (not is_machine_in_alert):
        await self.__put_new_machine_in_alert_state(
            iot_data.machine,
            iot_data.datatype,
            iot_data.timestamp,
            iot_data.alert_type)
    if (not is_alert_value) and is_machine_in_alert and machine_sensor_in_alert:
        await self.__remove_machine_sensor_from_alert_state(
            machine_in_alert,
            iot_data.datatype,
            iot_data.timestamp)
\end{Verbatim}

The method \texttt{\_\_put\_new\_machine\_sensor\_in\_alert\_state} is a private asynchronous method that is responsible for adding a new sensor to the alert state for a specific machine. It receives two parameters: \texttt{machine\_in\_alert}, which is an instance of the \texttt{MachinesSensorAlert} class representing the machine in question, and \texttt{sensor\_type}, which is an instance of the \texttt{Datatype} type, shown in \ref{subsubsec:constantes}, representing the type of sensor that should be put on alert.

\begin{Verbatim}[fontsize=\small, baselinestretch=0.8]
class MachinesSensorAlert(BaseModel):
    id: PyObjectId = Field(default_factory=PyObjectId, alias="_id")
    machine:str = Field(...)
    sensors:list[str] = Field([])
    alert_type:str = Field(...)
    start_time:datetime = Field(...)
    sensors_historical:list[str] = Field([])
    is_in_alert:bool = Field(True)
    end_time:Optional[datetime|None] = Field(None)
    read_by:Optional[list[str]] = Field([])
\end{Verbatim}

It is important to highlight that within this instance that is kept in memory, the \texttt{read\_by} attribute is not filled. This happens because this attribute is used to control the users who marked the notification as read, and thus identify notifications read by the user. Therefore, this attribute is filled only in the database, by the notifications module of the \gls{API}, shown in \ref{subsec:modules}.

The first step performed by this method is to identify the position (or index) of the machine within the \texttt{machines\_alert} list using the \texttt{index} method. Once the index is obtained, the sensor type is added to the machine's alert state sensor list, represented by the \texttt{sensors} attribute. In addition, this sensor is also added to the machine's alert state sensor history, indicated by the \texttt{sensors\_historical} attribute. Finally, the updated machine (with the new sensor added to its alert and history lists) is reinserted into the main \texttt{machines\_alert} list at the same position identified earlier.

This method ensures that whenever a new sensor enters an alert state for a machine that already had a sensor in alert, the relevant information is properly updated and kept in memory, allowing real-time monitoring of the alert conditions of all monitored machines.

\begin{Verbatim}[fontsize=\small, baselinestretch=0.8]
async def __put_new_machine_sensor_in_alert_state(
    self,
    machine_in_alert: MachinesSensorAlert,
    sensor_type:Datatype):
        index = self.machines_alert.index(machine_in_alert)
        machine_in_alert.sensors.append(sensor_type.value)
        machine_in_alert.sensors_historical.append(sensor_type.value)
        self.machines_alert[index] = machine_in_alert
\end{Verbatim}

The \texttt{\_\_put\_new\_machine\_in\_alert\_state} method is a private asynchronous method whose main function is to create and register a new alert state for a specific machine. This method is invoked when a machine enters an alert state for the first time, which means it is not yet present in the \texttt{machines\_alert} list of the class.

It receives four parameters: \texttt{machine\_name}, which is a string representing the machine's name; \texttt{sensor\_type}, which is an instance of the \texttt{Datatype} type, shown in \ref{subsubsec:constantes}, denoting the type of sensor that triggered the alert; \texttt{start\_time}, an instance of \texttt{datetime} indicating the start of the alert; and \texttt{alert\_type}, which is a string representing the type of alert.

Initially, the method creates a new instance of the \texttt{MachinesSensorAlert} class. This new instance represents the machine's alert state. The instance is initialized with the machine's name, the type of sensor that triggered the alert, a timestamp of the alert's start, and the type of alert. In addition, the machine is marked as being in an alert state through the \texttt{is\_in\_alert} attribute, which is set to \texttt{True}.

Finally, the machine's new alert state, represented by the newly created \texttt{MachinesSensorAlert} instance, is added to the \texttt{machines\_alert} list.

\begin{Verbatim}[fontsize=\small, baselinestretch=0.8]
async def __put_new_machine_in_alert_state(self,
    machine_name:str,
    sensor_type:Datatype,
    start_time:datetime,
    alert_type:str):
        new_machine_alert = MachinesSensorAlert(
            machine=machine_name,
            sensors=[sensor_type.value],
            sensors_historical=[sensor_type.value],
            is_in_alert=True,
            start_time=start_time,
            alert_type=alert_type)
        self.machines_alert.append(new_machine_alert)
\end{Verbatim}

The method \texttt{\_\_remove\_machine\_sensor\_from\_alert\_state} is a private asynchronous function designed to remove a specific sensor from a machine's alert state. It takes three parameters: \texttt{machine\_in\_alert}, which is an instance of the \texttt{MachinesSensorAlert} class representing the machine in question; \texttt{sensor\_to\_remove}, which is of the \texttt{Datatype} type \ref{subsubsec:constantes}, and identifies the sensor to be removed; and \texttt{end\_time}, an instance of \texttt{datetime} that indicates the moment when the sensor was removed from the alert state. Within this method, initially, the positions of the sensor and the machine are identified in the appropriate lists. The sensor is then removed from the machine's list of sensors in alert state. If, after removal, the machine no longer has sensors in alert state, it will be removed from the alert state, by calling the method \texttt{\_\_remove\_machine\_from\_alert} otherwise, only the sensor's state is updated, by calling another method, \texttt{\_\_remove\_sensor\_from\_alert\_state}.

\begin{Verbatim}[fontsize=\small, baselinestretch=0.8]
async def __remove_machine_sensor_from_alert_state(self,
    machine_in_alert: MachinesSensorAlert,
    sensor_to_remove:Datatype,
    end_time:datetime):
        index_of_machine = self.machines_alert.index(machine_in_alert)
        index_of_sensor = machine_in_alert.sensors.index(sensor_to_remove.value)
        machine_in_alert.sensors.pop(index_of_sensor)
        if len(machine_in_alert.sensors) == 0:
        await self.__remove_machine_from_alert(index_of_machine, end_time)
        else:
        await self.__remove_sensor_from_alert_state(index_of_machine,machine_in_alert)
\end{Verbatim}

The \texttt{\_\_remove\_machine\_from\_alert} method is another private asynchronous function, which is responsible for completely removing a machine from the alert state. It accepts two parameters: \texttt{index\_of\_machine}, the index of the machine in question in the list, and \texttt{end\_time}, the time when the machine was removed from the alert. Within this method, the machine is first marked as not being on alert and then it is removed from the \texttt{machines\_alert} list. The machine is then stored in the database with a record of its final state and the end time. Finally, a notification is sent through a websocket to inform the user interface about the change in the machine's state. The details of how the notification is sent are explained in \ref{subsubsec:WebSocketImplement}.

\begin{Verbatim}[fontsize=\small, baselinestretch=0.8]
async def __remove_machine_from_alert(self,
    index_of_machine:int,
    end_time:datetime):
    machine_in_alert = self.machines_alert[index_of_machine]
    machine_in_alert.is_in_alert = False
    machineNotification = self.machines_alert.pop(index_of_machine)
    machineNotification.end_time = end_time
    await self.iot_database.insert_one(
    NOTIFICATION_DATABASE,
    IOT_MACHINE_ALERTS,
    machineNotification.to_bson())
    await self.websocket.send_notification(machineNotification)    
\end{Verbatim}

The method \texttt{\_\_remove\_sensor\_from\_alert\_state} is a simple asynchronous function that updates the state of a machine's sensor in the alert list. It receives two parameters: \texttt{index\_of\_machine}, which is the index of the machine in the \texttt{machines\_alert} list, and \texttt{machine\_alert\_updated}, which is the updated instance of the machine in alert. Essentially, this method replaces the existing machine in the list with the updated object provided as a parameter by the \texttt{\_\_remove\_machine\_sensor\_from\_alert\_state} method.

\begin{Verbatim}[fontsize=\small, baselinestretch=0.8]
async def __remove_sensor_from_alert_state(self,
    index_of_machine:int,
    machine_alert_updated:MachinesSensorAlert):
    self.machines_alert[index_of_machine] = machine_alert_updated
\end{Verbatim}
\section[Implementation of the data processing module]{Implementation of the data processing module}\label{sec:ImplModuloProcessamento}

As explained in ~\ref{subsec:moduloProcessamento}, the data processing module reads the raw data from the system, applies the \texttt{boxplot} calculation, and stores the result in the database.

\subsection{Scheduling for periodic execution}
The data processing needs to occur periodically, in this case it was initially defined once a day. To execute the processing function call once a day, the \texttt{schedule} library was used \cite{scheduleDocs}. With this library, the execution of the data aggregation function was scheduled for every day at midnight. An infinite loop was created to keep the code running, checking whether the function should be executed or not.

\begin{Verbatim}[fontsize=\small, baselinestretch=0.8]
schedule.every().day.at("00:00").do(aggregation_init)
print(datetime.now(), flush=True)
while True:
    schedule.run_pending()
    time.sleep(1)
\end{Verbatim}

\subsection{Identifying the origin of the data}
Within this structure, it is necessary to identify the correct collections from which the data should be retrieved before processing. This identification begins with the function \texttt{get\_tuples\_with\_raw\_data\_collections\_and\_processed\_collections()}. This function, as the name suggests, is responsible for retrieving tuples relating the raw data collections with their respective processed collections. It iterates over all types of sensors, represented by the enumerator \texttt{Datatype}, explained in \ref{subsubsec:constantes}, and for each type of sensor, the respective raw and processed data collections are identified, resulting in a list of tuples.

It is important to highlight that the names of the collections are retrieved by the helper functions, explained in \ref{subsubsec:helpers}.

\begin{Verbatim}[fontsize=\small, baselinestretch=0.8]
def get_tuples_with_raw_data_collections_and_processed_collections():
    result:list[tuple] = []
    for sensor_type in Datatype:
        raw_collection = sensor_name_to_raw_data_collection(
        sensor_type)
        processed_collection = sensor_name_to_processed_collection(
        sensor_type)
        result.append((raw_collection,processed_collection))
    return result
\end{Verbatim}

To initiate the data processing, the function \texttt{aggregation\_init()} is called, first obtaining the list of tuples that relate the raw data collections with the processed ones. After retrieving this list, it initializes an asynchronous loop, whose aim is to execute an aggregation function until its completion. This asynchronous design is necessary to ensure that the processing can make asynchronous function calls, given that this module is separate from the \gls{API}.

\begin{Verbatim}[fontsize=\small, baselinestretch=0.8]
def aggregation_init():
    tuples_list = 
    get_tuples_with_raw_data_collections_and_processed_collections()
    loop = asyncio.new_event_loop()
    loop.run_until_complete(aggregation(tuples_list))
    loop.close()
\end{Verbatim}

In this way, the origin of the data is identified, and where they should be inserted after being processed. This information is passed to the aggregation function so that it can be executed for any stored data.

\subsection{Starting the aggregation}
Once the data source is defined through the identified collections, the data aggregation phase is initiated. The function responsible for this task is \texttt{aggregation()}, which accepts a list of tuples representing the sensor collections.

\begin{Verbatim}[fontsize=\small, baselinestretch=0.8]
async def aggregation(sensors_collection_list:list[tuple]):
    database = BaseDB()
    for collection_tuple in sensors_collection_list:
        (raw_data_collection, processed_data_collection) = collection_tuple
        machine_list = await database.read_machines_list(raw_data_collection)
        for machine in machine_list:
            await aggregate_data(
                database,
                raw_data_collection,
                processed_data_collection,
                machine)
\end{Verbatim}

Within this function, firstly, a database instance is initialized using the \texttt{BaseDB()} class, explained in \ref{subsubsec:DatabaseImpl}. Then, the function iterates over each tuple in the provided list. For each tuple, the raw and processed data collections are extracted. Using the raw data collection as a reference, a reading of the list of machines associated with this collection is made through the \texttt{read\_machines\_list()} method.

\begin{Verbatim}[fontsize=\small, baselinestretch=0.8]
async def read_machines_list(self, collection:str):
    temp_client = self.client
    return await temp_client[IOT_DATABASE][collection].distinct('machine')
\end{Verbatim}

For each identified machine, the data is then aggregated. The function \texttt{aggregate\_data()} is called, passing the database, the raw data collection, the processed data collection, and the specific machine in question as arguments. This function, in turn, is responsible for effectively aggregating the machine's data, transforming raw data into processed data that will be stored in the respective processed data collection.

\subsubsection{Searching for data to be aggregated}
Initially, a \texttt{query} is generated using the \texttt{get\_aggregation\_query()} function, which uses the information from the aggregated collection and the machine in question. With this \texttt{query}, the raw data is then read from the raw data collection using the \texttt{read\_raw\_data()} method.

The \texttt{get\_aggregation\_query()} function is responsible for generating the query that searches for the information to be aggregated by the processing module. Its goal is that only the raw data not yet processed are considered for aggregation, optimizing the process and avoiding unnecessary reprocessing.

This function requires an instance of the database, the name of the collection where the aggregated data is stored, and the specific machine for which the aggregation is needed.

\begin{Verbatim}[fontsize=\small, baselinestretch=0.8]
async def get_aggregation_query(
    database:BaseDB,
    collection:str,
    machine:str)->dict:
    field_to_aggregate = "more_recent_register"
    more_recent_processed_data:BoxPlotData|None = 
    await database.read_more_recent_data(
        collection,
        machine,
        field_to_aggregate)
    
    if more_recent_processed_data is None:
        return __build_query_with_limit_of_data(machine) 
    else:
        return __build_query_with_range_of_data(
        more_recent_processed_data,
        machine,
        field_to_aggregate)
\end{Verbatim}

The construction of the \textit{query} uses two important constants. \texttt{MAX\_VALUE\_BY\_PERIOD} stores the maximum number of records that can be read for aggregation, which in this case is the equivalent amount to 24 hours of reading considering the arrival of data every second, or 86400 records. Meanwhile, \texttt{AGGREGATION\_PERIOD\_IN\_HOURS} stores the number of hours between one aggregation and another, in this case 24 hours, being consistent with the previous constant.

Initially, the field \texttt{more\_recent\_register} is set as the attribute to be searched for. The function \texttt{read\_more\_recente\_data()} is then called to obtain the most recent processed data for the machine and collection in question.

\begin{Verbatim}[fontsize=\small, baselinestretch=0.8]
async def read_more_recente_data(self,
    collection:str,
    machine:str,
    date_time_field:str):
    try:
        temp_client = self.client
        cursor = temp_client[IOT_PROCESSED_DATA][collection]
            .find({"machine":machine})
            .sort([(date_time_field,pymongo.DESCENDING)])
        result:list = await cursor.to_list(None)
        return result[0] if len(result)!= 0 else None
    except Exception as ex:
        print(ex)
        raise ex
\end{Verbatim}

If no recently processed data is found, the \textit{query} is built using the \texttt{\_\_build\_query\_with\_limit\_of\_data()} function. This function simply limits the amount of data retrieved to \texttt{MAX\_VALUE\_BY\_PERIOD} and searches for records that match the specified machine.

\begin{Verbatim}[fontsize=\small, baselinestretch=0.8]
def __build_query_with_limit_of_data(machine:str)->dict:
    return {"limit":MAX_VALUE_BY_PERIOD,"query":{"machine":machine}}
\end{Verbatim}


However, if recent processed data is found, which is expected, the function to be used is \texttt{\_\_build\_query\_with\_range\_of\_data()}. This function considers the most recent processed record and calculates a time range (\texttt{date\_limit\_to\_process\_data}) by adding the aggregation period, defined by \texttt{AGGREGATION\_PERIOD\_IN\_HOURS}, to the date of this most recent record. The generated query searches for records with timestamps within this time range and that match the specified machine, with a maximum limit of
records defined by \texttt{MAX\_VALUE\_BY\_PERIOD}.

\begin{Verbatim}[fontsize=\small, baselinestretch=0.8]
def __build_query_with_range_of_data(more_recent_processed_data:BoxPlotData,
machine:str,
field_to_aggregate:str)->dict:
    date_of_more_recent:datetime = 
        more_recent_processed_data[field_to_aggregate]
    date_limit_to_process_data = date_of_more_recent + 
        timedelta(hours = AGGREGATION_PERIOD_IN_HOURS)
    return {
        "query":{
            "timestamp": {
                "$gt": date_of_more_recent,
                "$lte": date_limit_to_process_data
            },
            "machine":machine
        },
        "limit":MAX_VALUE_BY_PERIOD
    }
\end{Verbatim}

\subsubsection{BoxPlot Calculation}
With the query assembled, the data is retrieved using the \texttt{read\_raw\_data} function.

\begin{Verbatim}[fontsize=\small, baselinestretch=0.8]
async def read_raw_data(self, collection:str, query:dict):
    try:
        temp_client = self.client
        cursor = temp_client[IOT_DATABASE][collection].find(
            query["query"])
            .sort([("timestamp",pymongo.ASCENDING)])
            .limit(query["limit"])
        return await cursor.to_list(None)
    except Exception as ex:
        print(ex)
        raise ex
\end{Verbatim}

The amount of data retrieved is calculated and, if this amount exceeds a predefined minimum value \texttt{MINIMUM\_DATA\_TO\_AGGREGATE}, the aggregation proceeds. If the minimum amount is not reached, the recursive function is terminated, concluding the processing of the data from that collection.

\texttt{MINIMUM\_DATA\_TO\_AGGREGATE} is a constant value defined as 100, which ensures that there are enough data to be aggregated, avoiding the aggregation of a small amount of data, which could compromise the analysis.

After the query search, a \texttt{logger} object is initialized to keep records of the aggregation process.

In this implementation, the log is used only to display information on the console, but a future implementation may add a more comprehensive form of logs, as detailed in \ref{subsubsec:futurelogs}.

\begin{Verbatim}[fontsize=\small, baselinestretch=0.8]
class Logger(metaclass=Singleton):
    async def store_aggregation_log(self,
        box_plot_data:BoxPlotData,
        collection:str):
            ...
            
    async def not_aggregated_data(self, amount:int, collection:str):
        ...
\end{Verbatim}

The raw data read from the database is converted into a DataFrame, from the \texttt{pandas} library \cite{pandasDocs} (a python language library used for data manipulation), after which the relevant aggregated data is calculated using the \texttt{calc\_box\_plot()} function. This function returns the data in a structured form suitable for graphical representations, such as a box plot.

To perform the calculation, various functions from the pandas library are used, such as \texttt{median}, \texttt{quartile}, \texttt{mean}, and \texttt{shape}, which facilitate the understanding and execution of the calculation.

\begin{Verbatim}[fontsize=\small, baselinestretch=0.8]
def calc_box_plot(df:pd.DataFrame, machine:str):
    values = df["value"]

    median = values.median()
    mean = values.mean()

    Q1 = values.quantile(.25)
    Q3 = values.quantile(.75)

    IIQ = Q3 - Q1

    lower_quartile = Q1 - 1.5 * IIQ
    upper_quartile = Q3 + 1.5 * IIQ
    
    selection = (df["value"]>=lower_quartile) & (df["value"]<=upper_quartile)

    values_selected = values[selection]

    mean_with_selection = values_selected.mean()

    df['timestamp'] = pd.to_datetime(df['timestamp'])
    date_of_more_recent:datetime|str = df['timestamp'].max()

    amount_of_data = df.shape[0]

    box_plot = BoxPlotData()

    box_plot.more_recent_register:datetime = date_of_more_recent

    box_plot.lower_quartile=lower_quartile
    box_plot.upper_quartile=upper_quartile
    box_plot.median=median
    box_plot.mean=mean
    box_plot.mean_with_selection=mean_with_selection
    box_plot.q1=Q1
    box_plot.q3=Q3
    box_plot.amount_of_data=amount_of_data
    box_plot.machine=machine

    return box_plot
\end{Verbatim}

In this function, the received dataframe contains a series of values that will be used to calculate the components of the Box Plot. First, the median and mean values of the data are determined. The quartiles Q1 (first quartile) and Q3 (third quartile) are calculated using the \texttt{quantile()} function from the pandas library. From these quartiles, the \gls{IQR} is determined as the difference between Q3 and Q1.

To identify the outlier values, the lower and upper limits are calculated. The lower limit is obtained by subtracting \(1.5 \times \texttt{IQR}\) from Q1 and the upper limit is obtained by adding \(1.5 \times \texttt{IQR}\) to Q3. Subsequently, a selection of values that are between the lower and upper limits is made. The mean of these selected values is then calculated, resulting in \texttt{mean\_with\_selection}.

The function also takes care of converting the \texttt{timestamp} column to datetime type and identifying the most recent \textit{timestamp}, which will be crucial for assembling searches in the following aggregations.

With all the calculated values, a \texttt{BoxPlotData} object is instantiated and populated with the Box Plot components, along with additional information, such as the total number of data and the corresponding machine.


\subsubsection{Recording of processed data}
After the entire process described, the data is converted into JSON format and inserted into the collection of aggregated data by the \texttt{insert\_processed\_data} function.

\begin{Verbatim}[fontsize=\small, baselinestretch=0.8]
async def insert_processed_data(self, collection:str, data):
    try:
        temp_client = self.client
        await temp_client[IOT_PROCESSED_DATA][collection]
            .insert_one(data)
    except Exception as ex:
        print(ex)
        raise ex
\end{Verbatim}

After a successful insertion, the function \texttt{aggregate\_data()} is recursively called, ensuring that all relevant raw data are aggregated.

However, if the amount of raw data does not reach the minimum limit, the function records this occurrence using the \texttt{not\_aggregated\_data()} method, indicating that the data were not aggregated due to their lack, and ends the recursion.


\section[API Implementation]{API Implementation}\label{sec:api}
As explained in the section about the architecture ~\ref{subsec:apiArchitecture}, the \gls{API} has a division by modules, and each module follows a predefined structure, with a controller layer, responsible for receiving \gls{HTTP} requests, a service layer, responsible for handling data and business rules, and a repository layer, responsible for managing access to the database of that module.

In addition, the \gls{API} also has parts that are common to all modules. The infrastructure that has the function of providing a database access interface, a web socket message sending access interface, authentication means, and the thread manager. The common codes, which store constants, common functions that need to be standardized, and data models.
Therefore, this section will address each of these parts, first going through the common parts of the system and then showing how a complete module was developed.

\subsection{Initialization}\label{subsec:main}
The system initialization occurs through the \texttt{main.py} file, which serves as the entry point to initialize the \gls{API} and the data processing module.

The \texttt{FastAPI} library \cite{fastapiDocs} is used to create the main application, here referred to as \texttt{app}. The \texttt{CORSMiddleware} middleware is added to the \texttt{FastAPI} application, allowing for comprehensive \gls{CORS} (Cross-Origin Resource Sharing) configuration. This configuration allows the \gls{API} to be accessed from different origins.

The import of the \texttt{API\_data\_layer} module not only incorporates the routes related to this module, but also initializes the data receiving module. This implies that the initialization of this module occurs simultaneously with the loading of the \gls{API}, but on a separate thread, as detailed in ~\ref{sec:ImplModuloProcessamento}.

For metadata management, an instance of the \texttt{MetadataRepository} is created during initialization. This component is essential for loading the metadata that are used in different parts of the system, such as constants and alarm parameters.

The \gls{API} routes are then included in the main application through the \texttt{include\_router} method for different modules, such as authentication, API analysis, data layer, notifications, and users.

Additionally, the WebSocket is mounted at the root of the application through the \texttt{socketio\_app} object, enabling real-time communication between the server and the clients. The implementation of the websocket connection can be seen in ~\ref{subsubsec:WebSocketImplement}.

The execution of the file concludes with the initialization of the \texttt{Uvicorn} server \cite{uvicornOfficialDocs}, setting the host and port for listening. \texttt{Uvicorn} is an \gls{ASGI} server that serves as the interface between the application code and the web server. It is responsible for hosting the \texttt{FastAPI} application and listening for incoming connections on the specified host and port. The choice of this server was based on the recommendation from the FastAPI documentation \cite{fastapiTutorial}.

\begin{verbatim}
from fastapi import FastAPI
import uvicorn
from fastapi.middleware.cors import CORSMiddleware
from src.infrastructure.database.metadata.metadata_repository import (
MetadataRepository)
from src.modules.api_analytics import api_analytics_router
from src.modules.api_data_layer import api_data_layer_router
from src.modules.notifications import notification_module_router
from src.modules.user import user_module_router, auth_router
from src.infrastructure.websocket import socketio_app
from src.infrastructure.websocket import socket_dispacher
from dotenv import load_dotenv

load_dotenv()
MetadataRepository()
socket_dispacher
app = FastAPI()
app.add_middleware(
    CORSMiddleware,
    allow_origins=["*"],
    allow_credentials=True,
    allow_methods=["*"],
    allow_headers=["*"]
)
@app.get("/")
async def health_check():
    return {
        "Status":"OK",
        "Message":"Access /docs for more information"    
    }
app.include_router(auth_router)
app.include_router(api_analytics_router)
app.include_router(api_data_layer_router)
app.include_router(notification_module_router)
app.include_router(user_module_router)

app.mount("/",socketio_app)

if __name__ == "__main__":
    uvicorn.run(app, host="0.0.0.0", port=8000)
\end{verbatim}

\subsection{Infrastructure}\label{subsec:infra}

The infrastructure is composed of 4 sub-modules, Authentication, WebSocket, Database Connection, and Thread Management.

\subsubsection{Authentication}\label{subsubsec:auth}
The system's authentication implementation was based on the official FastAPI documentation \cite{fastapiSecurity}, therefore a token-based approach was adopted using \gls{JWT}. The \gls{JWT} is a widely accepted standard for securely transmitting information between parties. The structure of a \gls{JWT} is encoded and can be verified to ensure that the data has not been altered during transmission.

The entry point for authentication is the \texttt{o\_auth2\_password\_bearer}, an instance of OAuth2PasswordBearer that is designed to obtain the token from the request header. The \texttt{auth\_middleware} method was defined as an asynchronous middleware, which depends on this bearer token. This middleware is used in the controllers to verify whether the received request has permission to access the information or not.

Within this middleware, the \texttt{decode\_jwt\_token} function is invoked to decode and validate the provided \gls{JWT} token.

\begin{verbatim}
o_auth2_password_bearer = OAuth2PasswordBearer(tokenUrl="/user/login")

async def auth_middleware(token:str = Depends(o_auth2_password_bearer))-> TokenPayload:
    try:
        result = decode_jwt_token(token)
        if result.status:
            return result.data
        else:
            raise HTTPException(status_code=401, detail=result.exception.message)
    except JWSError as jwt_err:
        print(jwt_err)
        raise HTTPException(status_code=401, detail=Unauthorized().message)
\end{verbatim}

The function \texttt{decode\_jwt\_token} receives a token as an argument and attempts to decode it using the specified secret key and algorithm. If the token is successfully decoded and is of the type "\texttt{access\_token}". Otherwise, different types of exceptions can be raised, for example, if the token has expired or if there is some error in the \gls{JWT} operations.

\begin{verbatim}
def decode_jwt_token(token:str)->Result[TokenPayload|None]:
    try:
        token_dict = jwt.decode(token,key=SECRET_KEY, algorithms=ALGORITHM)
        if token_dict["type"] != "access_token":
            return Result(status=False, exception=WrongTokenType(), data=None)
        token_payload = TokenPayload(**token_dict)
        return Result(status=True, data=token_payload, exception=None)
    
    except ExpiredSignatureError as invalid_token:
        return Result(status=False, exception=Unauthorized(exception=invalid_token), data=None)

    except (JWSError, JOSEError, JWTError, JWEError) as ex:
        return Result(status=False, exception=GenericException(message="Authorization error!",exception=ex), data=None)

    except Exception as ex:
        print(ex)
        raise ex
\end{verbatim}

The model class for the token payload is as follows:
\begin{verbatim}
class TokenPayload(BaseModel):
    name:str
    exp:int|None = None
    sub:str|None = None
    user_id:str
    type:str = "access_token"
\end{verbatim}

The \texttt{AuthService} class is where the main authentication logic is implemented. This class follows the \textit{Singleton} pattern to ensure that only one instance is created and used throughout the program execution.

Within \texttt{AuthService}, the \texttt{verify\_password} method is used to check if a provided password matches an encrypted password, while the \texttt{hash\_password} is responsible for encrypting a provided password.

The \texttt{create\_user\_tokens} method generates a pair of tokens (access and refresh) for a user, where the access token is valid for 4 hours and the refresh token for 168 hours. The refresh token is especially important to allow users to obtain new access tokens without having to enter their credentials again. If the access token expires, the refresh token can be used to obtain a new pair of tokens, using the \texttt{get\_new\_user\_tokens} method. It is important to note that the frontend does not yet make use of the refresh token, leaving this functionality for a future implementation.

\begin{verbatim}
class AuthService(metaclass=Singleton):
    def __init__(self):
        self.__database__ = MongoDB()
        self.__user_repository = UserRepository()

        self.__ACCESS_TOKEN_EXPIRE_HOURS__ = 4
        self.__REFRESH_TOKEN_EXPIRE_HOURS__ = 168
        self.__SECRET_KEY__ = SECRET_KEY
        self.__ALGORITHM__ = ALGORITHM
        self.__pwd_context__ = CryptContext(
            schemes=["bcrypt"],
            deprecated="auto")

    def verify_password(self,plain_text_password:str, hashed_password:str):
        return self.__pwd_context__.verify(plain_text_password,hashed_password)

    def hash_password(self,password:str):
        return self.__pwd_context__.hash(password)
    
    async def create_user_tokens(self, user:User)->tuple[str,str]:
        payload = TokenPayload(name=user.name,user_id=str(user.id))
        access_token = self.__create_bearer_token(
            user_id=user.id,
            data=payload.__dict__,
            expire_hours=self.__ACCESS_TOKEN_EXPIRE_HOURS__)
        refresh_token = await self.__create_refresh_token(str(user.id))
        return (access_token, refresh_token)
        
    async def get_new_user_tokens(self,
        refresh_token:str) -> Result[tuple[str, str]]:
        result = self.__decode_jwt_refresh_token(refresh_token)
        if not result.status:
            return Result(status=False, exception=result.exception, data=None)
        
        token_payload = result.data
        is_valid = await self.__check_if_refresh_token_is_valid(token_payload)
        if not is_valid:
            return Result(status=False, data=None, exception=Unauthorized())
        
        user = await self.__user_repository.read_user_by_id(token_payload.user)
        payload = TokenPayload(name=user.name,user_id=str(user.id))
        new_access_token = self.__create_bearer_token(
            user_id=token_payload.user,
            data=payload.__dict__,
            expire_hours=self.__ACCESS_TOKEN_EXPIRE_HOURS__)

        return Result(status=True, data=(new_access_token, refresh_token), exception=None)
\end{verbatim}

The methods \texttt{\_\_create\_bearer\_token} and \texttt{\_\_decode\_token} are auxiliary functions used to create, decode, and verify tokens, respectively.

\begin{verbatim}
def __create_bearer_token(self,user_id:int, data:dict, expire_hours):
    data_to_enconde = data.copy()
    expire = datetime.now()+timedelta(hours=expire_hours)
    data_to_enconde["exp"] = expire
    data_to_enconde["sub"] = str(user_id)
    return jwt.encode(
        claims=data_to_enconde,
        key=self.__SECRET_KEY__,
        algorithm=self.__ALGORITHM__)

def __decode_token(self,token:str)->dict:
    return jwt.decode(
        token,
        key=self.__SECRET_KEY__,
        algorithms=self.__ALGORITHM__)
    
\end{verbatim}

\subsubsection{WebSocket}\label{subsubsec:WebSocketImplement}
The implementation of the connection via WebSocket was done using the \texttt{socket.io} library \cite{socketIoDocs}, which has various ready-to-use features that facilitate the management of Web Socket connections, such as the creation of rooms for firing notifications.

The structure adopted for managing Web Socket connections was designed in such a way that the client must make a websocket request to the root \textit{endpoint} of the \gls{API} to be registered in a specific virtual room. After token validation and successful completion of this request, the client begins to receive all messages directed to the room in which it was registered.

The current implementation contemplates only one room, specifically intended for sending notifications related to the operation of the machines. This room is identified by the identifier \texttt{NOTIFICATION\_ROOM}. This constant stores the value "\texttt{Notification}", which is the name of the room to be connected, stored along with the system constants described in \ref{subsubsec:constantes}.

The asynchronous mode \gls{ASGI} selected for the server creation, and the allowed origins for \textit{CORS} are set as empty.

\begin{verbatim}
socket_io_server = AsyncServer(async_mode="asgi",
    cors_allowed_origins=[])

socketio_app = ASGIApp(socketio_server=socket_io_server,
    socketio_path="")

socket_dispacher = WebSocketDispatcher(socket_io_server)

@socket_io_server.event
async def connect(sid, environ, auth):
    token = auth["Authorization"]
    await auth_middleware(token)
    socket_io_server.enter_room(sid, NOTIFICATION_ROOM)
\end{verbatim}

The \texttt{socket\_io\_server} object is responsible for managing the \textit{WebSocket} communication, while the \texttt{socketio\_app} creates an \gls{ASGI} application that interacts with the \textit{WebSocket} server, and is added to the FastAPI server. Additionally, an instance of the \texttt{WebSocketDispatcher} class was created to facilitate the sending of notifications through the \textit{WebSocket}. This configuration is done at system initialization, explained in \ref{subsec:main}

In the connection event, named \texttt{connect}, a client is automatically added to the \texttt{NOTIFICATION\_ROOM}.

Finally, the \texttt{WebSocketDispatcher} class has a \texttt{send\_notification} method, which is used to send notifications. When calling this method, the notification is converted to the JSON format and sent to all clients in the \texttt{NOTIFICATION\_ROOM} through the \texttt{emit} method.

\begin{verbatim}
class WebSocketDispatcher:
    def __init__(self,socket_io_server: AsyncServer):
        self.__socket_io_server = socket_io_server
    
    async def send_notification(self,
        machine_sensor_notification:MachinesSensorAlert):
        machine_sensors_dict = machine_sensor_notification.to_json()
        await self.__socket_io_server.emit(
            NOTIFICATION_ROOM,
            machine_sensors_dict,
            room=NOTIFICATION_ROOM)
\end{verbatim}

The \texttt{WebSocketDispatcher} class is used by the data receiving module, detailed in ~\ref{subsec:checkDataReceived}, to trigger notifications when an improper operation of the machines is identified.

\subsubsection{Thread Management for Asynchronous Tasks}\label{subsubsec:ThreadManager}
In the system architecture, the need to perform tasks concurrently, without blocking the execution of the \gls{API}, was identified. These tasks are the execution of the data reception module, and the checking of the system's metadata. Both tasks must be executed in parallel with the execution of the \gls{API}, without influencing its execution. Therefore, to achieve this goal, a thread manager, called \texttt{ThreadManager}, was implemented.

The \texttt{ThreadManager} class is designed following the Singleton pattern, ensuring that only one instance is created, thus avoiding conflicts or redundancies in thread management. A list called \texttt{threads} is initialized to store all created threads, while an asynchronous event loop, assigned to the \texttt{loop} variable, is created using the \texttt{asyncio} library \cite{pythonAsyncio}.

The \texttt{start\_async\_thread} method was introduced to facilitate the creation and management of asynchronous tasks. This method accepts an asynchronous function, \texttt{func}, as an argument and performs the following operations:

\begin{enumerate}
    \item An internal function \texttt{start\_function} is defined. This function is responsible for starting the execution of the asynchronous task.
    \item Within \texttt{start\_function}, the boolean variable \texttt{isSeted} is checked to determine if the event loop has already been set up. Otherwise, the event loop is set up and the asynchronous task is executed until completion through the \texttt{run\_until\_complete} method.
    \item If the event loop is already set up (\texttt{isSeted = True}), the asynchronous task is simply added to the existing loop using \texttt{create\_task}.
    \item Finally, a new thread is created with \texttt{start\_function} as the target and added to the \texttt{threads} list. The thread is then started, executing the asynchronous task.
\end{enumerate}

\begin{verbatim}
import asyncio
from threading import Thread

class ThreadManager(metaclass=Singleton):
    def __init__(self):
        self.threads = []
        self.loop = asyncio.new_event_loop()
        self.isSeted = False

    def start_async_thread(self,func):
        def start_function():
            if not self.isSeted:
                asyncio.set_event_loop(self.loop)
                self.isSeted = True
                self.loop.run_until_complete(func())
            else:
                self.loop.create_task(func())

        new_thread = Thread(target = start_function)
        self.threads.append(new_thread)
        new_thread.start()
\end{verbatim}

This implementation allows the execution of multiple asynchronous tasks in parallel, each in its own thread, all managed by the same asynchronous event loop.

\subsubsection{Database}\label{subsubsec:DatabaseImpl}
In the process of implementing the system, to establish an efficient connection with the database, the \texttt{motor} library \cite{motorDocs} was adopted as the mechanism.

At the heart of the connection strategy is a base class, named \texttt{BaseDB}, which is responsible not only for establishing the connection with MongoDB, but also for defining a series of basic operations for the manipulation of stored data. The structure of this class is presented below:

\begin{verbatim}
import motor
class BaseDB:
    def __init__(self):
        self.client = motor.motor_tornado.MotorClient(url, port)
\end{verbatim}

Some of the fundamental operations implemented by \texttt{BaseDB} include:

\begin{itemize}
    \item \texttt{insert\_one}: Receives as parameters the corresponding \textit{database} and \textit{collection} in text format, and the \textit{data} to be inserted. It inserts a document into the specified collection.
    
    \item \texttt{insert\_many}: Receives as parameters the corresponding \textit{database} and \textit{collection} in text format, and the \textit{data} containing several documents to be inserted. It inserts several documents into the specified collection.

    \item \texttt{read\_data\_with\_pagination}: Receives as parameters the \textit{database}, the \textit{collection}, the \textit{query}, the \textit{page\_number}, the \textit{limit}, the \textit{sort\_descending\_field}, and the \textit{projection}. Retrieves data with pagination, allowing for a more organized reading.
    
    \item \texttt{read\_data\_with\_limit}: Receives as parameters the \textit{database}, the \textit{collection}, the \textit{query}, and the \textit{limit}. Reads data with a predefined limit of returned documents.
    
    \item \texttt{read\_data}: Receives as parameters the \textit{database}, the \textit{collection}, and the \textit{query}. Performs a simple data reading based on a query.
    
    \item \texttt{get\_distinct\_property}: Receives as parameters the \textit{database}, the \textit{collection}, and the \textit{property}. It obtains distinct properties from a collection, checking all the present documents.
    
    \item \texttt{list\_collections\_by\_db}: Receives as a parameter the \textit{database}. It lists all the collections present in a specific database.
    
    \item \texttt{add\_item\_into\_lists\_by\_filter}: Receives as parameters the \textit{database}, the \textit{collection}, the \textit{filter}, the \textit{list\_properties}, and the \textit{new\_data}. It adds an item into specific lists based on a filter.
    
    \item \texttt{update\_item}: Receives as parameters the \textit{database}, the \textit{collection}, the \textit{data} to be updated, and the \textit{filter}. It updates a specific document.
    
    \item \texttt{update\_many\_items}: Receives as parameters the \textit{database}, the \textit{collection}, the \textit{data} to be updated, and the \textit{filter}. It updates several documents that meet a filter.
    
    \item \texttt{count\_documents}: Receives as parameters the \textit{database}, the \textit{collection}, and the \textit{query}. It counts the number of documents in a collection that meet a query.
    
    \item \texttt{get\_data\_between\_dates}: Receives as parameters the \textit{database}, the \textit{collection}, and the \textit{query}. Retrieves data between two specific dates.
\end{itemize}

With the access base established, other classes were developed, inherited from \texttt{BaseDB}, to meet specific system contexts. These classes follow the singleton pattern, which ensures that only one instance of the connection is created for a specific context, optimizing resource management. An example is the \texttt{MongoDBIOT} class intended for the data reception module:

\begin{verbatim}
class MongoDBIOT(BaseDB, metaclass=Singleton):
    def __init__(self):
        super().__init__()
\end{verbatim}

Similar classes, following the same format, were created for other contexts, such as database access through the \gls{API}, ensuring an organized and efficient structure for connection and data manipulation.

\subsection{Common Files}\label{subsec:commum}

Within the structure of the \gls{API}, a folder named \texttt{common} was implemented with the aim of centralizing reusable components, covering multiple modules and layers. This organization was established to maximize development efficiency and code consistency.

\subsubsection{Data Models}\label{subsubsec:dataModel}
The data models section in the \texttt{common} folder houses various classes that define the structure of the used data. Classes specifying users and sensor data are present, and they make use of the \texttt{Pydantic} library to define the models and create data validations.

\texttt{Pydantic} \cite{pydanticDocs} is a data validation library that adds static typing in Python to validate that the received data matches a certain format or schema. When used for the construction of the \gls{API}, \texttt{Pydantic} contributes to the automatic and consistent verification of data sent through \gls{HTTP} requests, and manipulations performed in the database. This approach reduces the need for manual coding for data validations, thus speeding up development time and increasing code robustness.

To illustrate, we have the NotificationSensorResponse class, which is responsible for defining the data model that is returned when the \gls{API} receives a request for a specific user's notifications. The use of \texttt{BaseModel} in the inheritance system, from \texttt{Pydantic}, necessary to define the return types within FastAPI, is highlighted. In addition, the \texttt{Field} function, also from \texttt{Pydantic}, is used to indicate with three points that it is a mandatory attribute to be informed in the class construction.

\begin{verbatim}
class NotificationSensorResponse(BaseModel):
    data:list[dict] = Field(...)
    total_count:int = Field(...)
\end{verbatim}

Other important data models are the exception classes and a class called \texttt{Result}, responsible for data traffic between the different layers of the application, shown in the implementation of the \gls{API} module in \ref{subsec:modules}.
Example of a class that defines a system exception.

\begin{verbatim}
class CustomBaseException(Exception):
    def __init__(self, message:str, exception, *args: object) -> None:
        self.message = message,
        self.exception = exception
        super().__init__(*args)

class GenericException(CustomBaseException):
    def __init__(self,
        message:str =   "An error has occurred",
        exception = None) -> None:
        super().__init__(message, exception)
\end{verbatim}


Class \texttt{Result} used for communication between layers of the system modules. The \texttt{TypeVar} is used to indicate a generic type for the \texttt{data} attribute of the class.

\begin{verbatim}
from typing import TypeVar, Generic
from src.common.models.exceptions.unauthorized import CustomBaseException

T = TypeVar('T')
class Result(Generic[T]):
    def __init__(self, status:bool, data:T|None, exception:CustomBaseException|None):
        self.status = status
        self.data = data
        self.exception = exception
\end{verbatim}

The use of the \texttt{Singleton} class is also highlighted, used when there is a need to ensure that only one instance of a certain class will be used.

\begin{verbatim}
class Singleton(type):
    _instances = {}
    def __call__(cls, *args, **kwargs):
        if cls not in cls._instances:
            cls._instances[cls] = super(Singleton, cls).__call__(*args, **kwargs)
        return cls._instances[cls]
\end{verbatim}

Finally, the \texttt{PyObjectId} class is highlighted, used to define a type for the ID attribute of classes that represent database models. The methods defined for this class are used internally by \texttt{FastAPI} and \texttt{Pydantic}.

This class is necessary so that the ID can be correctly converted to text, and returned in the requests. Furthermore, the use of this class enables the manipulation of the ID when necessary, leaving the management of unique identifiers to the application and not to the database.

\begin{verbatim}
class PyObjectId(ObjectId):
    @classmethod
    def __get_validators__(cls):
        yield cls.validate

    @classmethod
    def validate(cls, v):
        if not ObjectId.is_valid(v):
            raise ValueError("Invalid object id")
        return ObjectId(v)

    @classmethod
    def __modify_schema__(cls, field_schema):
        field_schema.update(type="string")
\end{verbatim}

\subsubsection{Helper Functions}\label{subsubsec:helpers}

The section of \textit{helper} functions in the \texttt{common} folder was built to contain methods that are common in different modules of the system, avoiding code duplication and standardizing operation.

An example of these functions is the conversion of \texttt{Datatype}, shown in the section on constants \ref{subsubsec:constantes}, to MongoDB collections. The following code illustrates one of these processes:

\begin{verbatim}
def sensor_name_to_processed_collection(
    sensor_name:Datatype)->str:
    sensor = (sensor_name.name).capitalize()
    return IOT_AGGREGATION_COLLECTION.replace("NAME", sensor)
\end{verbatim}

These functions are used to determine the correct names of the collections according to the type of data to be processed and manipulated by the modules. Functions such as \texttt{sensor\_name\_to\_processed\_collection} and \texttt{sensor\_name\_to\_raw\_data\_collection} convert between sensor names and collection names.

Additionally, a series of functions was developed to map the processed collection name back to the corresponding \texttt{Datatype}, shown in the section on constants \ref{subsubsec:constantes}, ensuring safer and more consistent data manipulation.

\subsubsection{Constants}\label{subsubsec:constantes}
Finally, the constants section stores a series of fixed values that are used in various parts of the system. This includes database names, alert types, websocket rooms, and others as needed.

Among the constants, \texttt{DataType} stands out, which is an enum that standardizes the types of data that can be received from the sensors. This standardization is employed in various modules to ensure that data is received, processed, stored, and returned in a consistent and correct manner.

\begin{verbatim}
from enum import Enum

class Datatype(Enum):
    PRESSURE = "PRESSURE"
    TEMPERATURE = "TEMPERATURE"
    VOLTAGE = "VOLTAGE"
    CURRENT = "CURRENT"
    SPEED = "SPEED"
    ACCELERATION = "ACCELERATION"
    DISTANCE = "DISTANCE"
    HUMIDITY = "HUMIDITY"
    FORCE = "FORCE"
    PRODUCTION_COUNTER = "PRODUCTION_COUNTER"
\end{verbatim}

\subsection{Modules}\label{subsec:modules}
The \gls{API} was structured into modules, each one responsible for managing a certain context. The developed modules are listed below.

\begin{enumerate}
    \item \textbf{IOT Analytics}: Module responsible for managing access to sensor information, both real-time data via stream, and the information processed by the data processing module, explained in ~\ref{sec:ImplModuloProcessamento}.
    \item \textbf{Notifications}: Module responsible for managing access to notifications.
    \item \textbf{User}: Module responsible for managing the system users' information, and performing the authentication explained in ~\ref{subsubsec:auth}.
    \item \textbf{Downtime Analytics}: Module used to provide test data on machine downtime. This module feeds the screen that displays information about machine downtime.
\end{enumerate}

Each module followed a pattern of having one layer for receiving \gls{HTTP} requests, the \texttt{controller}, another to handle the request according to business rules, the \texttt{service}, and the \texttt{repository}, to provide methods for accessing and manipulating information in the database.

\subsubsection{Controller}\label{subsubsec:controller}
The controller has the function of receiving \gls{HTTP} requests, sending the received information to the service layer, and making the appropriate return, with the formatted information, and correct \gls{HTTP} code.

To exemplify the operation of the controller, a specific example will be presented. The following code snippet represents the router of the \gls{API} for the \gls{IoT} sensor data:

\begin{verbatim}
iot_data_router = APIRouter(tags=["IOT Data"], dependencies=[Depends(auth_middleware)])

service = ServiceIOT()

@iot_data_router.get("/realtime")
async def real_time_iot(): 
    def get_real_time_data():
        sensor = SensorValue()
        while True:
            time.sleep(1)
            lista_json = [machine.to_json() for machine in sensor.machine_list]
            last_data = json.dumps(lista_json)
            yield bytes(last_data, "utf-8")

    return StreamingResponse(
        get_real_time_data(),
        media_type="application/octet-stream")
\end{verbatim}

In this example, the controller uses the \texttt{APIRouter} to create routes associated with IoT data. An authentication middleware is applied as a dependency, as shown in the authentication detail in \ref{subsubsec:auth}, ensuring that only authenticated users can access these routes. In this case, by applying the middleware in the creation of the \texttt{iot\_data\_router}, it is guaranteed that all \texttt{endpoints} created from it require authentication.

Below, an instance of the service to be used by the \texttt{endpoints} to respond to the received requests is created.

The \texttt{/realtime} route is designated to provide real-time data. An internal function, \texttt{get\_real\_time\_data}, is responsible for collecting this data from the \texttt{SensorValue}, explained in \ref{sec:Implementation of the data reception module}. In this endpoint, the \gls{API} takes the current values and returns a \texttt{StreamingResponse}, which sends real-time data as a continuous stream.

The other routes, such as \texttt{/machines\_in\_sensor} and \texttt{/graph\_info}, operate in a similar way, but with different responsibilities. They make calls to the \texttt{ServiceIOT} instance to retrieve specific information and return it to the client. If an exception or error occurs, an \texttt{HTTPException} is thrown with an appropriate HTTP status code and a detailed error message.

In these two endpoints, it is important to highlight the specification of the \texttt{response\_model}, so that automatic validations by the framework occur before the data is returned, ensuring consistency in the returned data.

\begin{verbatim}
@iot_data_router.get("/machines_in_sensor", response_model=list[MachinesSensor])
async def get_machines_in_sensor():
    result = await service.get_all_machines_processed_info()
    if result.status:
        return result.data
    
    raise HTTPException(status_code=500, detail=result.exception.message)

@iot_data_router.get("/graph_info", response_model=list[ProcessedData])

\begin{verbatim}
async def get_graph_info(machine:str, sensor:Datatype, initial_date:datetime, end_date:datetime):
    result = await service.get_processed_data(
        machine=machine,
        data_type=sensor,
        initial_date=initial_date,
        end_date=end_date)
    
    if result.status:
        return result.data
    
    raise HTTPException(status_code=500, detail=result.exception.message)
\end{verbatim}

\subsubsection{Service}\label{subsubsec:service}
The service layer is used to apply the appropriate rules before returning the data to the control layer ~\ref{subsubsec:controller}. It can access the repository layer to read or write data, and it should return the data to the control layer using the \texttt{Result} class, shown in section ~\ref{subsec:commum}, so that the result of the processing can be correctly identified, and the appropriate return can be given to the client who made the request.

For a deeper understanding, a specific example of the service layer implementation follows below:

\begin{verbatim}
class ServiceIOT:
    def __init__(self):
        self.__repository = RepositoryIOT()
        self.__database__ = MongoDB()
        self.appMetadata = MetadataRepository()
            
    async def get_processed_data(self,
        machine:str,
        data_type:Datatype,
        initial_date:datetime,
        end_date:datetime)-> Result[list[ProcessedData]]:
        try:
            alert_parameter = await self.appMetadata.get_sensor_alert_value(
            data_type)
            processed_data = await self.__repository.read_iot_processed_data__(
                machine=machine,
                datatype=data_type,
                initial_date=initial_date,
                end_date=end_date,
                sort_by_field="more_recent_register")
            
            for data in processed_data:
                data["alert_parameter"] = alert_parameter
            return Result[list[MachinesSensor]](
                status=True,
                data=processed_data,
                exception=None)
        except Exception as ex:
            return Result[list[MachinesSensor]](
                status=False,
                data=None,
                exception=GenericException(exception=ex))
\end{verbatim}

In this implementation, the \texttt{ServiceIOT} class is initialized with instances of \texttt{RepositoryIOT} and \texttt{MetadataRepository}, allowing the service to access the corresponding repository and metadata layers.

The \texttt{get\_processed\_data} method is used to obtain processed data based on various parameters such as machine, data type, and date range. Initially, an alert value is retrieved from the sensor metadata for the provided data type. Subsequently, the processed data is read from the repository. For each retrieved data entry, the alert value is added as a new field. The method returns a \texttt{Result} object encapsulating these data. The \texttt{Result} object is detailed in ~\ref{subsec:commum}.

The \texttt{get\_all\_machines\_processed\_info} method retrieves information about all processed machines and sensors. It iterates through the processed data collections, aggregating machine and sensor information. In case of success, it returns a \texttt{Result} object containing a list of machines and their corresponding sensors.

\begin{verbatim}
async def get_all_machines_processed_info(
    self)-> Result[list[MachinesSensor]]:
    try:
        collection_list = 
            await self.__repository.get_processed_data_collections()
        machine_sensors: list[MachinesSensor] = []
        for collection in collection_list:
            machine_list = await self.__repository
                .get_distinct_machines_by_collection(collection)
            sensor = processed_collection_to_sensor_name(collection)

            for machine in machine_list:
                matching_sensor = 
                next((data for data in machine_sensors 
                    if data.machine == machine), None)
                if matching_sensor:
                    matching_sensor.sensors.append(sensor)
                else:
                    machine_sensors.append(MachinesSensor(
                        machine=machine,
                        sensors=[sensor]))
        
        return Result[list[MachinesSensor]](
            status=True,
            data=machine_sensors,
            exception=None)
    except Exception as ex:
        print(ex)
        return Result[bool](
            status=False,
            data=None,
            exception=GenericException(exception=ex)) 
\end{verbatim}

The last method, \texttt{read\_raw\_data\_by\_id}, is used to read raw data based on an identifier and data type. It accesses the corresponding repository to retrieve the data and returns a \texttt{Result} object containing these data or an exception, if applicable.
\begin{verbatim}
async def read_raw_data_by_id(self,
    raw_data_id:str,
    datatype:Datatype) -> Result:
    try:
        collection = sensor_name_to_raw_data_collection(datatype)
        result = await self.__repository.read_raw_data(collection,raw_data_id)
        return Result(status=True, data=result, exception=None) 
    except Exception as ex:
        return Result[bool](
            status=False,
            data=None,
            exception=GenericException(exception=ex)) 

\end{verbatim}
This implementation exemplifies how the service layer interacts with the repository layers and accesses metadata, and how it prepares the data to be sent back to the controller, thus ensuring a cohesive and efficient data flow through the various layers of the application.

\subsubsection{Respository}\label{subsubsec:repository}
Finally, the repository layer is responsible for accessing the database and performing read and write operations as needed. This layer initiates a connection with the database, and its methods use the base methods defined by the database connection infrastructure, in ~\ref{subsubsec:DatabaseImpl}, to perform operations according to the context of that module.

The following code provides an example of the implementation of this layer:

\begin{verbatim}
class RepositoryIOT:
    def __init__(self):
        self.__database__ = MongoDB()
    
    async def get_processed_data_collections(self):
        collection_list = 
            await self.__database__.list_collections_by_db(
            IOT_PROCESSED_DATA)
    
        return collection_list
        
    async def get_distinct_machines_by_collection(self, collection:str):
        machine_list = 
            await self.__database__.get_distinct_property(
                IOT_PROCESSED_DATA,
                collection,
                "machine")

        return machine_list
    
    async def read_raw_data(self, collection:str, raw_data_id:str):
        result = await self.__database__.read_data(
            IOT_DATABASE,
            collection,
            {"_id":ObjectId(raw_data_id)})

        return result
    
    async def read_iot_processed_data__(self, machine:str, datatype:Datatype, 
        initial_date:datetime, end_date:datetime, 
        sort_by_field:str) -> list[ProcessedData]:
        
        try:
            collection = sensor_name_to_processed_collection(datatype)
            query = {
                "machine":machine,
                sort_by_field:{
                    "$gte":initial_date,
                    "$lte":end_date
                }
            }
            result = await self.__database__.get_data_between_dates(
                IOT_PROCESSED_DATA,
                collection,query)
            return result
        except Exception as ex:
            print(ex)
            raise ex
\end{verbatim}

Upon initializing the \texttt{RepositoryIOT} class, a connection with MongoDB is instantiated. The \texttt{get\_processed\_data\_collections} method is used to list all the processed data collections from the \texttt{IOT\_PROCESSED\_DATA} database. This method makes a direct call to the collection listing method provided by the \texttt{MongoDB} class.

The \texttt{get\_distinct\_machines\_by\_collection} method is responsible for retrieving a list of distinct machines for a given collection. It does this through the \texttt{get\_distinct\_property} method of the database.

The \texttt{read\_raw\_data} method is used to read raw data from a specific collection, using the data ID as a search parameter. It makes a call to the \texttt{read\_data} method of the \texttt{MongoDB} class, providing the necessary parameters for data reading.

Finally, the \texttt{read\_iot\_processed\_data\_\_} method is used to read processed data based on various criteria such as machine, data type, and date range. A query is built for this purpose and passed to the \texttt{get\_data\_between\_dates} method of the \texttt{MongoDB} class.

Each of these methods assists in maintaining a clear separation of responsibilities, allowing the service layer to maintain a strict focus on business logic, while the repository layer manages database operations.

\section[Implementação do frontend]{Implementação do frontend}\label{sec:implFront}

A implementação da interface de usuário, foi desenvolvida de acordo com a arquitetura exposta na seção \ref{sec:archFront}. O desenvolvimento do \textit{frontend} é segmentado em várias partes, que incluem a organização das páginas do sistema conforme a estrutura pré definida do Next.js \cite{nextjsDocs}, o gerenciamento de dados acessados pelos componentes, a configuração para acesso externo, e a construção dos componentes individuais.

É relevante notar que, para manter a conformidade com as melhores práticas e simplificar o desenvolvimento, as configurações padrão do Next.js foram mantidas.

\subsection{Paginas do sistema}\label{subsec:}
Por se tratar de um framework, o NextJs tem uma estrutura pré definida para criação das páginas do sistema assim como suas rotas \cite{nextjsDefiningRoutes}. Dentro dos arquivos do framework, a pasta \textit{pages} é utilizada para armazenar cada uma das páginas do sistema, sendo cada arquivo uma página, e o nome do arquivo sendo a rota para acesso. A configuração das páginas acorre por arquivos com nomes específicos, no caso \textit{\_app.tsx} e \textit{\_document.tsx}.

\subsubsection{Paginas de configuração}\label{subsec:configPage}
Em relação as páginas de configuração temos primeiro a \textit{\_app.tsx}. Esse arquivo tem a responsabilidade de configurar e gerenciar contextos, estilização global e a localização de datas, ou seja, aspectos globais de toda a aplicação.

O código começa pela importação de diversos módulos e bibliotecas, o que inclui contextos específicos como \texttt{OpenContext} e \texttt{PrivateContext}, que são explicados melhor em \ref{subsubsec:contextCreation}, e o suporte para localização de datas com \texttt{AdapterDayjs} \cite{dayJsInstallation}.

O \texttt{LocalizationProvider} e \texttt{AdapterDayjs} são de bibliotecas que têm o objetivo de fornecer funcionalidades de localização e formatação de datas. O \texttt{LocalizationProvider} atua como um encapsulador para o sistema de datas, permitindo a integração com diferentes bibliotecas de gerenciamento de datas. Neste caso, o \texttt{AdapterDayjs} é utilizado como o adaptador para a biblioteca Day.js, permitindo que as datas sejam manipuladas e formatadas de maneira eficiente e compatível com diversos locais geográficos e formatos. Com essas bibliotecas fica mais fácil gerenciar datas para a construção do filtro do dashboard que gerencia o período de datas exibido nos gráficos, explicados em \ref{sec:histicalGraphs}. 

O tipo \texttt{NextPageWithLayout} foi definido para enriquecer as propriedades da página com informações sobre o \textit{layout}. Isso permite que cada página tenha um \textit{layout} personalizado se necessário, oferecendo grande flexibilidade no design da interface.

A função principal \texttt{App}, que recebe \texttt{Component} e \texttt{pageProps} como argumentos, é responsável por configurar o \textit{layout} e renderizar os componentes da página. A lógica dentro desta função verifica a rota atual usando \texttt{useRouter} \cite{nextjsUseRouter} para determinar se o usuário está na página de login.

O conteúdo é então encapsulado dentro dos contextos relevantes. Se o usuário estiver na página de login, apenas o \texttt{OpenContext} é aplicado. Para todas as outras páginas, o \texttt{PrivateContext} é adicionalmente aplicado, garantindo que as informações sensíveis sejam acessadas apenas por usuários autenticados. Os contextos utilizados são detalhados em \ref{subsec:contextApi}.

Dentro do \texttt{LocalizationProvider}, o adaptador \texttt{AdapterDayjs} é utilizado para fornecer funcionalidades de localização de datas, tornando o aplicativo mais versátil em diferentes locais.


\begin{verbatim}
import {OpenContext, PrivateContext} from '@/context'
import '@/styles/globals.css'
import { LocalizationProvider } from '@mui/x-date-pickers'
import { AdapterDayjs } from '@mui/x-date-pickers/AdapterDayjs'
import { NextPage } from 'next'
import type { AppProps } from 'next/app'
import { useRouter } from 'next/router'
import { ReactElement, ReactNode } from 'react'

export type NextPageWithLayout<P = {}, IP = P> = NextPage<P, IP> & {
    getLayout?: (page: ReactElement) => ReactNode
}

type AppPropsWithLayout = AppProps & {
    Component: NextPageWithLayout
}

export default function App({ Component, pageProps }: 
    AppPropsWithLayout) {

    const getLayout = Component.getLayout || ((page) => page)
    const router = useRouter()
    const isLoginPage = router.pathname === "/"

    const componentWithProps = <Component {...pageProps} /> 

    return getLayout(
    <LocalizationProvider dateAdapter={AdapterDayjs}>
        <OpenContext>

        {isLoginPage?
            <>{componentWithProps}</>
            :<PrivateContext>
                {componentWithProps}
            </PrivateContext>
        }

        </OpenContext>  
    </LocalizationProvider>
    )
}
\end{verbatim}

Embora seja um arquivo mais simples comparado ao \texttt{\_app.tsx}, o \texttt{\_document.tsx} tem a responsabilidade de definir da estrutura HTML global da aplicação.

No arquivo, foram importados os componentes \texttt{Html}, \texttt{Head}, \texttt{Main}, e \texttt{NextScript} da biblioteca \texttt{next/document}. Estes componentes são utilizados para criar a estrutura básica da página HTML dentro do NextJs.

O componente \texttt{Html} é utilizado para encapsular todo o conteúdo HTML e inclui o atributo \texttt{lang="en"}, o qual define o idioma da página como inglês. O componente \texttt{Head} \cite{nextjsHeadComponent} é empregado para adicionar elementos no cabeçalho da página HTML. Neste caso, o título da página é definido como 'Dashboard'.

O corpo da página HTML é composto pelos componentes \texttt{Main} e \texttt{NextScript}. O \texttt{Main} é o local onde o conteúdo principal da página é inserido, enquanto o \texttt{NextScript} é responsável por incluir os scripts necessários para o funcionamento do Next.js.

Vale destacar que o \texttt{\_document.tsx} não tem acesso a características específicas da página como os métodos \texttt{getInitialProps} \cite{nextjsInitialProps}, \texttt{getStaticProps} \cite{nextjsGetStaticProps}, ou \texttt{getServerSideProps} \cite{nextjsGetServerSideProps} (funções do NextJs para carregado de dados do lado do servidor). Isso implica que este arquivo é ideal para configurações que são comuns em todas as páginas e não requerem informações dinâmicas.

\begin{verbatim}
import { Html, Head, Main, NextScript } from 'next/document'

export default function Document() {
  return (
    <Html lang="en">
      <Head title='Dashboard'/>
      <body>
        <Main />
        <NextScript />
      </body>
    </Html>
  )
}
\end{verbatim}

\subsubsection{Paginas do sistema}\label{subsec:}
As páginas do sistema se dividem em dois tipos, privadas e publica, sendo que publica é apenas a página de login. Essa página pública está no arquivo \texttt{index.tsx}, sendo a rota raiz do sistema. 


Neste arquivo, apenas configurações \texttt{meta} e o componente \texttt{Login} são invocados. O elemento \texttt{Head} \cite{nextjsHeadComponent} é utilizado para definir configurações globais do HTML, como o título da página e metadados. 

O componente \texttt{Login} é chamado dentro da tag \texttt{main}, que serve como o conteúdo principal da página. Esta abordagem de design mantém a página \texttt{index.tsx} enxuta, transferindo a maior parte da lógica e da apresentação visual para o componente \texttt{Login}. Este é um exemplo do princípio de separação de interesses, onde cada arquivo ou componente tem uma única responsabilidade claramente definida.

\begin{verbatim}
export default function Home() {
    return (
      <>
        <Head>
          <title>Catraport Dashboard</title>
          <meta name="description" 
            content="Generated by create next app" />
          <meta name="viewport" 
            content="width=device-width,
            initial-scale=1" />
          <link rel="icon" href="/favicon.ico" />
        </Head>
        <main>
          <Login/>
        </main>
      </>
    )
  }
\end{verbatim}

As outras páginas do sistema se encontram dentro da pasta \texttt{dashboard}, que também está dentro da pasta \texttt{pages}. Isso implica que todas as páginas dentro dessa pasta devem ser acessados na rota \texttt{/dashboard} \cite{nextjsDefiningRoutes}.

Dentro do dashboard, existe a páginas principal, em \texttt{/index.tsx}, com a página do dashboard que exibe os dados em tempo real e os gráficos com os dados históricos processados. As funcionalidades dessa página sáo detalhadas em \ref{cap:functions}.

Este arquivo segue à mesma lógica de design observada na página de login, mantendo a separação entre as configurações da página e a lógica dos componentes invocados.

O componente \texttt{Dashboard} se baseia em composição, delegando diversas responsabilidades a componentes individuais. O componente \texttt{DashboardLayout} é utilizado como um contêiner que define a estrutura global da página, oferecendo um layout consistente também para as outras páginas do dashboard. Dentro deste componente, vários outros são chamados para realizar funções específicas.

O \texttt{DashboardHeader} é responsável pela exibição do cabeçalho da página, fornecendo o acesso aos filtros para visualização das informações. Segue-se o componente \texttt{SensorsValues}, que é designado para mostrar os valores dos sensores em tempo real.

Um elemento \texttt{Divider}, da biblioteca \texttt{Material UI 5} \cite{muiDocs} é inserido para fornecer uma separação visual entre as diferentes seções da página. Por fim, o componente \texttt{SensorsGraphs} é invocado para exibir gráficos relacionados aos dados históricos dos sensores de forma agregada.

\begin{verbatim}
export default function Dashboard() {
    return (
        <DashboardLayout>
            <DashboardHeader/>
            <SensorsValues/>
            <Divider/>
            <SensorsGraphs/>
        </DashboardLayout>
    )
}
\end{verbatim}

As outras páginas do dashboard também foram construídas usando a lógica de composição demostrada e utilizando o mesmo componente base para o layout, \texttt{DashboardLayout}. Essas páginas são:
\begin{enumerate}
    \item \textbf{Maintenace}: Responsável por exibir os dados de paragem das maquinas em forma de gráficos para demostrar a visualização dessas informações dentro do sistema, \ref{sec:downtime}.
    \item \textbf{Profile}: Responsável por exibir as informações do usuário que está logado no sistema, assim como permitir realizar alterações nos dados, \ref{sec:profile}.
\end{enumerate}


\subsection{Gerencia dos dados do sistema}\label{subsec:contextApi}
Um dos pontos importantes no desenvolvimento do frontend é o gerenciamento de estados globais, que são informações ou comportamentos compartilhados entre componentes não relacionados. Dentro desse projeto, a \texttt{Context API} \cite{reactCreateContext} foi utilizada para este propósito, senso uma solução nativa do React que se destaca pela sua facilidade de implementação e utilização. Esse recurso permite que dados sejam passados de forma eficiente em toda a árvore de componentes, eliminando a necessidade de passar manualmente propriedades através de níveis intermediários.

Não apenas estados de dados, mas também estados comportamentais, como o estado de login do usuário e o estado do menu lateral (aberto ou fechado), foram gerenciados por meio do \texttt{Context API}. A camada de dados foi construída de tal forma que todas as informações necessárias para o funcionamento do sistema, que dependem de um valor global, foram incluídas.

Os contextos criados para o gerenciamento de estados são os seguintes:
\begin{enumerate}
    \item \texttt{SnackbarContext}: Utilizado para o gerenciamento de mensagens e alertas no sistema, facilitando o acesso a função que exibe alertas e sua configuração em todo o sistema.
    \item \texttt{AuthContext}: Encarregado de gerenciar o estado de autenticação do usuário, \ref{subsubsec:auth}.
    \item \texttt{ThemeContext}: Responsável pela gestão do tema visual da aplicação \cite{muiDefaultTheme}.
    \item \texttt{NotificationContext}: Utilizado para o gerenciamento, leitura e recebimento de notificações no sistema.
    \item \texttt{LegacyContext}: Gerencia informações de parada das máquinas, após serem lidas do backend.
    \item \texttt{IotContext}: Utilizado para o gerenciamento de estados relacionados aos dispositivos IoT no sistema.
    \item \texttt{DrawerContext}: Encarregado de gerenciar o estado do menu lateral (aberto ou fechado), e disponibilizar em todas as páginas do sistema que utilizam o menu lateral.
\end{enumerate}

Cada contexto foi concebido com uma função específica, de modo a permitir uma separação clara das responsabilidades. Utilizando a arquitetura mostra em X, esse modelo de gerenciamento de dados no frontend fica simples e escalável, como especificado nos requisitos, em \ref{sec:req}.

\subsubsection{Gerencia de diferentes contextos}\label{subsubsec:difContexts}
Para tratar da complexidade gerada pela variedade de contextos necessários no sistema, optou-se por instanciar esses contextos por meio de componentes específicos. Esses componentes foram projetados para encapsular diferentes grupos de contextos, de acordo com as necessidades de acesso.

Dois componentes principais foram desenvolvidos: \texttt{OpenContext} e \texttt{PrivateContext}. O primeiro é responsável por instanciar os contextos que estão disponíveis para qualquer indivíduo que acessar o sistema. O segundo é encarregado de instanciar contextos que só podem ser acessados por usuários autenticados. A criação de um contexto é exemplificado melhor em ~\ref{subsubsec:contextCreation}.

Esses componentes são usados no arquivo \texttt{\_\_app.tsx}, que é o arquivo de configuração inicial do Next.js, como explicado em ~\ref{subsec:configPage}, de modo a tornar os contextos acessíveis em toda a árvore de componentes da aplicação.

A seguir, é apresentado o código que ilustra como esses componentes foram implementados:

\begin{verbatim}
interface Props{
    children:React.ReactNode
}

function OpenContext({children}:Props){
    return (
        <SnackbarContextProvider>
            <AuthContextProvider>
                <ThemeContextProvider>
                        {children}
                </ThemeContextProvider>
            </AuthContextProvider>
        </SnackbarContextProvider>
    )
}

function PrivateContext({children}:Props){
    return(
        <>
            <NotificationProvider>
                <LegacyContext>
                    <IotContext>
                        <DrawerContextProvider>
                            {children}
                        </DrawerContextProvider>
                    </IotContext>
                </LegacyContext>
            </NotificationProvider>
        </> 
    )
}
\end{verbatim}

Desta forma, os contextos são adequadamente isolados e gerenciados, garantindo que os dados e funcionalidades corretos estejam disponíveis para os usuários, de acordo com seu nível de acesso.


\subsubsection{Criação de um contexto}\label{subsubsec:contextCreation}
Para a criação de um contexto, primeiro é necessário a definição correta dos tipos, mandatório devido ao uso do \texttt{Typescript}. Para cada contexto, uma interface de propriedades (\texttt{Props}) e um valor padrão (\texttt{DEFAULT\_VALUE}) são criados. 

Para mostrar a criação de um contexto dentro do projeto será usado como exemplo  o contexto \texttt{DrawerContext}, responsável por gerenciar o estado do drawer da aplicação. O código a seguir exemplifica como este contexto foi criado:

\begin{verbatim}
import { createContext, useContext, useState } from "react"

interface Props {
    open:boolean
    setOpen:React.Dispatch<React.SetStateAction<boolean>>
}

const DEFAULT_VALUE = {
    open:false,
    setOpen:()=>{}
}

const DrawerContext = createContext<Props>(DEFAULT_VALUE)

function DrawerContextProvider({ children }:{children:React.ReactNode}){
    const [open, setOpen] = useState<boolean>(false)

    return (
        <DrawerContext.Provider value={{open,setOpen}}>
            {children}
        </DrawerContext.Provider>
    )
}

export default function useDrawer(){
    return useContext(DrawerContext)
}

export {DrawerContextProvider};
\end{verbatim}

Neste exemplo, o contexto \texttt{DrawerContext} é criado utilizando a função \texttt{createContext} do React \cite{reactCreateContext}. A interface \texttt{Props} define os tipos para o estado aberto do drawer (\texttt{open}) e a função para definir esse estado (\texttt{setOpen}). Um valor padrão (\texttt{DEFAULT\_VALUE}) é estabelecido para inicializar o contexto. 

O componente \texttt{DrawerContextProvider} utiliza o estado React local para gerenciar o valor do estado \texttt{open}. Este valor e a função \texttt{setOpen} são então disponibilizados para todos os componentes filhos por meio do \texttt{DrawerContext.Provider}. Dessa forma, o componente \texttt{DrawerContextProvider} é usado no gerenciamento dos contexto, como explicado em ~\ref{subsubsec:difContexts}, para tornar acessível toda a informação para a árvore de componentes inteira.

A função \texttt{useDrawer} é uma função personalizada que facilita o acesso ao contexto \texttt{DrawerContext} em qualquer parte da aplicação. Dentro do react, funções com o 'use' na frente são denominadas \texttt{hooks}, como pode ser visto na documentação oficial em \cite{reactHooksReference}. 

Dessa forma, o contexto foi criado e pode ser utilizado para gerenciar o estado de aberto e fechado do menu lateral em toda a aplicação.

Esse modelo de criação de criação de contextos se repete para todos os outros contextos, com a adição do acesso ao backend, que é explicado em ~\ref{subsec:api_access}.

\subsection{Acesso externo}\label{subsec:api_access}
Para a interação com dados armazenados e gerenciados pelo backend, foi estabelecida uma camada de acesso externo no frontend. Essa camada serve como um ponto centralizado para todas as requisições de rede e é necessário para a leitura e manipulação de dados que estão fora do escopo do frontend, portanto lida com as requisições HTTP, conexão WebSocket e recebimento de dados via stream.

Os diversos contextos criados no sistema, conforme descritos em ~\ref{subsubsec:contextCreation}, utilizam essa camada de acesso externo para carregar os dados necessários. Na inicialização de cada contexto, chamadas de função para esta camada são realizadas, se necessário. Essas chamadas são responsáveis por fazer requisições ao backend e por receber as informações retornadas.

No exemplo abaixo é feito o uso do hook \texttt{useEffect} \cite{reactUseEffect} para chamar uma função que acessa a camada de acesso externo para carregar dados referentes aos sensores, sendo eles os dados em tempo real, e dados dos gráficos, assim que o contexto é inicializado.

\begin{verbatim}
const fetchSensorData = useCallback(async()=>{
    await Promise.all([
        getRealTimeDataData(reciveRealTimeData),
        fetchGraphData(),
    ])
},[])

useEffect(()=>{
    fetchSensorData()
},[])
\end{verbatim}


\subsubsection{Requisições HTTP}\label{subsubsec:httpRequest}
A biblioteca \texttt{Axios} \cite{axiosIntro} foi empregada para facilitar a realização das requisições ao backend. Esta biblioteca proporciona uma interface simples e eficiente para criação de requisições \gls{HTTP}, e foi integrada nas funções da camada de acesso externo. Essa biblioteca possibilita uma configuração inicial para ser utilizada em todas as requisições realizadas.

Na configuração utilizada, é lido das variáveis de ambiente do sistema, o endereço do backend e a url para adicionar as configurações a rota base, para onde todas as requisição devem ir. Outra configuração aplicada é a adição do interceptor \cite{axiosInterceptors} no momento que a requisição é realizada, adicionando no header a configuração necessária para a autenticação, fazendo a leitura do access token do local storage \cite{mdnLocalStorage}, e adicionando no formato correto para leitura do backend em requisições que precisam de autenticação.

\begin{verbatim}
const baseUrl = process.env.NEXT_PUBLIC_API_URL
const apiRoute = process.env.NEXT_PUBLIC_API_ROUTE

const baseApi = axios.create({
  baseURL: `http://${baseUrl}${apiRoute}`
});

baseApi.interceptors.request.use(function (config) {
    let token = localStorage.getItem("access_token")
    if (token) {
      config.headers['Authorization'] = `Bearer ${token}`;
    }
    return config;
  }, function (error) {
    return Promise.reject(error);
});
\end{verbatim}

As funções dessa camada utiliza essa configuração base para realizar a busca dos dados no backend. Na função abaixo é exemplificado um dessas funções de acesso externo para buscar os dados de um determinado gráfico. A função utiliza a configuração base do \texttt{axios} junto com uma url especifica para acesso ao endpoint desejado para buscar as informações. Se o retorno estiver com o \texttt{status code} igual a 200 significa que a requsição foi bem sucedida, então os dados são retornados para o contexto que fez a chamada dessa função. O contexto recebe os dados e disponibiliza para toda a aplicação, como explicado em ~\ref{subsec:contextApi}.

\begin{verbatim}
async function getGraphData(
  machine:string,
  type:SensorType,
  startDate:Dayjs,
  endDate:Dayjs):Promise<Array<MachineGraphAggregateData>>{
  try{
    let url = "/iot/graph_info"+get_graph_query(
      machine,
      type,
      startDate,
      endDate)
    let response = await baseApi.get(url)
    if(response.status === 200){
      return response.data
    }
    throw("Erro to access graph - "+response.status)
  }catch(ex){
    throw("Erro to access graph - "+ex)
  }
}
\end{verbatim}


\subsubsection{Conexão WebSocket}\label{subsec:websocketConncetion}
Além das requisições \gls{HTTP}, a camada de acesso externo também gerencia a conexão WebSocket. Especificamente, o contexto de notificações faz uso dessa conexão para receber e gerir notificações em tempo real.

A gestão da conexão WebSocket é realizada por meio da classe \textit{CustomSocketConnection}. Esta classe é projetada como um singleton, garantindo que uma única instância seja criada e reutilizada em toda a aplicação. Ela é responsável por inicializar e manter o objeto \texttt{Socket}, que é parte da biblioteca \textit{socket.io-client} \cite{socketIoClientApi}.

\begin{verbatim}
class CustomSocketConnection{
  private static _instance:CustomSocketConnection|null = null
  private _socketio: Socket|null = null

  private constructor() {
      if (CustomSocketConnection._instance === null) {
          CustomSocketConnection._instance = this;
      }
  }

  public static getInstance(): CustomSocketConnection {
      if (!CustomSocketConnection._instance) {
          CustomSocketConnection._instance = new CustomSocketConnection();
      }
      return CustomSocketConnection._instance;
  }

  get socketio(){
    if(this._socketio===null){
      let token = localStorage.getItem("access_token")
      let url = process.env.NEXT_PUBLIC_API_URL??"localhost"
      this._socketio = io(url, { autoConnect: false, auth: { "Authorization": token } })
    }
    return this._socketio
  }
}
\end{verbatim}

O método \texttt{getInstance()} assegura que apenas uma instância da classe seja criada. Essa instância é armazenada como um atributo estático e é retornada sempre que solicitada. 

Uma instância da classe \textit{CustomSocketConnection} é utilizada no contexto de notificações. Através dessa instância, o contexto consegue receber mensagens do servidor, manipulá-las e, em seguida, disponibilizá-las para toda a aplicação. 

A classe também inclui um mecanismo de autenticação. O token de acesso é recuperado do armazenamento do local storage \cite{mdnLocalStorage} e é utilizado como parte do cabeçalho de autenticação durante o processo de conexão.

Dentro do contexto de notificações, o atributo Socket da classe é utilizada na inicialização do contexto para realizar a conexão. A instancia da classe é armazenada em uma constante, e em seguinda três funções são cadastradas em eventos de conexão, recebimento de notificação e disconexão. A função que trata o evento de receber uma nova notificação, envia o dado para uma função auxiliar que tem  o objetivo de analizar o dado recebido e atualizar a lista de notificações que aparecem na tela.

\begin{verbatim}
useEffect(() => {
  const socket = CustomSocketConnection.getInstance()
  function onConnect () {
      console.log(`Connected with id: ${socket.socketio.id}`);
      console.log(`Connection Status: ${socket.socketio.connected}`);
  }

  function newNotification(data:any){
      console.log("Socket Io Notification", data);
      checkNewNotification(data);
  }

  function disconnect(data:any) {
      console.log("Socket Io disconnect", data);
      console.log("Connection Status");
  }

  socket.socketio.on('connect', onConnect);
  socket.socketio.on('Notification', newNotification);
  socket.socketio.on('disconnect', disconnect);

  return () => {
      socket.socketio.off('connect', onConnect);
      socket.socketio.off('Notification', newNotification);
      socket.socketio.off('disconnect', disconnect);
  };
}, [notifications]);
\end{verbatim}

Dessa forma a camada de acesso externo disponibiliza um meio de conexão Web Socket para ser usado para receber novas notificações enquanto o usuário está conectado ao sistema.

\subsubsection{Recebendo dados via Stream}\label{subsec:streamData}

Em relação a camada de acesso externo, o último tipo de conexão é o recebimento de dados via stream. Este mecanismo permite a atualização constante dos dados recebidos, garantindo assim um fluxo contínuo de informações atualizadas para a aplicação. Especificamente, essa conexão é usada para disponibilizar os dados dos sensores que são recebidos no backend, como explicado em \ref{sec:Implementação do modulo de recebimento de dados}.

O recebimento de dados via stream é implementado usando a função \textit{fetch} \cite{mdnFetchAPI}, nativa do JavaScript. Essa função é responsável por realizar a requisição ao endpoint correspondente e obter o fluxo de dados em tempo real.

A função \texttt{readStream} é utilizada para interpretar os dados do stream. Essa função recebe o leitor do corpo da resposta, e retorna os dados convertidos para o formato \gls{json}.

\begin{verbatim}
async function readStream(
  reader:ReadableStreamDefaultReader<Uint8Array> | undefined) {
  let result = await reader?.read()
  if (!result?.done) {
      let value = result?.value
      if(value){
          const jsonData = parseBytesToJson(value)        
          return jsonData
      }
  }
  return false

}
\end{verbatim}

A função \texttt{getRealTimeDataData} é a responsável por iniciar o processo de streaming de dados. É nessa função que o contexto de dados dos sensores IoT faz a chamada e, consequentemente, recebe e envia os dados para a função recebida como parametro.

Como a função não faz o uso da estrtura do axios explicado em ~\ref{subsubsec:httpRequest}, é necessário realizar a configuração de autenticação da mesma que foi explicado anteriormente na estrutura.

Após realizar a requisição, o leitor dos dados é lido do corpo da resposta da requisição, e passado para a função \texttt{readStream}, explicada anteriormente, para ser interpretada e convertida para \gls{json}.

\begin{verbatim}
async function getRealTimeDataData(setData: UpdateDataFromStream) {
  try{
    let token = localStorage.getItem("access_token")
    let url = baseApi.getUri()
    let response = await fetch(`${url}/iot/realtime`,{headers:{
        Authorization: `Bearer ${token}`,
        'Content-Type': 'application/json',
    },});

    const reader = response?.body?.getReader();
    let result:Array<MachineRealTimeData>|boolean = await readStream(reader)
    if(typeof(result) === "boolean"){
        throw "Error to read stream data"
    }
    do{
      if (Array.isArray(result)) {
          const typedResult = result as MachineRealTimeData[];
          setData(typedResult);
      }
      result = await readStream(reader)
    }while(result!==false)
  }catch(ex){
    console.error(ex)
    throw ex
  }
}
\end{verbatim}

A função \texttt{getRealTimeDataData} fica sendo executada enquanto a conexão está aberta e não ocorre nenhum erro. O método \texttt{setData} é então chamado para enviar o resultado da leitura para o contexto que realizou a chamada da função.

Dentro do contexto de dados dos sensores IOT, a função \texttt{setData} passada como parâmetro, apenas atualiza o estado global para atualizar a informação em todos os componentes que fazem uso dela.

\subsection{Construção dos componentes}\label{subsec:componentization}
Na implementação do sistema, foi utilizada a lógica de construção de componentes \cite{reactFirstComponent} para compor a telas. A modularidade e reutilização de código são fatores críticos que motivam essa escolha. 

%TODO 33ref para algo sobre evitar a passagem de muitos dados na arvores do react
Os componentes acessam dados diretamente dos contextos, conforme discuto em \ref{subsec:contextApi}. Este método facilita a passagem de dados e permite que os componentes sejam mais específicos em sua função, além de evitar a passagem de muitas propriedades na árvore de componentes.

A base para os componentes menores foi extraída da biblioteca \gls{MUI5} \cite{muiDocs}. Isso inclui elementos como botões, contêineres, caixas de texto e outros. Um exemplo prático é o menu lateral nas páginas do dashboard, onde o componente Drawer do \gls{MUI5} foi empregado. 

Um componente que merece destaque é o \texttt{Grid} do \gls{MUI5} \cite{muiReactGrid}. A utilização do Grid permitiu uma organização espacial dos elementos da interface do usuário de maneira simples. Esse sistema de grid oferece uma abordagem flexível para alocar espaço, alinhar conteúdo e lidar com variações de tela, o que é especialmente útil em aplicações web com muitas informações e diferentes componentes em tela. No menu lateral por exemplo, foi utilizado o componente \texttt{Grid} para organizar as informações do menu e definir o posicionamento de acordo com o estado de aberto ou fechado.

\begin{verbatim}
<Grid 
  container
  direction="column"
  justifyContent="space-between"
  alignItems={open?"center":"start"}
  height={"97vh"}
>
  <Grid 
    container
    item
    direction="column"
    justifyContent="space-between"
    alignItems={open?"center":"start"}
  >
    // Conteúdo
  </Grid>
</Grid>
\end{verbatim}

O uso do Grid, portanto, contribuiu para a coesão do layout, fornecendo uma estrutura sólida sobre a qual outros componentes poderiam ser organizados de forma simples, e fácil de entender, cumprindo requisitos de facilitar a manutenção especificado em \ref{ssubec:reqNfuctional}.

Por outro lado, para a construção dos gráficos, a biblioteca Recharts foi utilizada. O gráfico mostrado no dashboard com a informações geradas pelo modulo de processamento, detalhado REF \ref{sec:ImplModuloProcessamento}, é composto por gráficos Scatter, Area, Bar e Line, permitindo assim uma análise multifacetada das informações geradas. 

Para componentes que necessitam de manipulação de datas, a biblioteca \texttt{Days Js} \cite{dayJsInstallation} foi integrada. Um caso de uso é o componente de filtro de data para exibição de gráficos, que também emprega o DatePicker \cite{muiDatePickerValidation} do \gls{MUI5} para uma interface de usuário mais intuitiva. A biblioteca Days Js facilita a manipulação da entrada e saída de dados de data.

\begin{verbatim}
  <DatePicker
  value={dateFilter.startDate}
  onChange={(value)=>onChangeDate(value, "startDate")}
  label="Data inicial"
  />
\end{verbatim}

\begin{verbatim}
const onChangeDate = (newDate:Dayjs|null, 
  dateField:"startDate"|"endDate")=>{
  if(newDate!==null){
    setIsDataUpdated(false)
    if(dateField==="endDate"){
      setDateFilter(oldValue=>({...oldValue,"endDate": newDate}))
    }else{
      setDateFilter(oldValue=>({...oldValue,"startDate": newDate}))
    }
  }
}
\end{verbatim}



\section[Adaptando a implementação para outros contextos]{Adaptando a implementação para outros contextos}
Discussão sobre a reutização do sistema para outros contextos....
- como fazer
- alterações necessarias
