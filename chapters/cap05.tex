\chapter{Implementação}\label{cap:test}

This chapter presents and describes the tests that were developed to check if the project fulfills the objectives and solves the problem described in Analysis/Methodology.

To better understand, the results of each test should be preceded by a description of the test and the expected results.

The work results are commented, including:

\begin{itemize}
	\item  What can be learned from the results?
	\item What could be done differently? 
	\item What was beyond initial objectives?
	\item What are the objectives there were not met and why?
\end{itemize}


\subsection[Implementação do banco de dados]{Implementação do banco de dados}
- Databases
- Collections

\subsection[Implementação da API]{Implementação da API}

- Uvicorn usado pelo FastAPI e sua forma asyncrona
- Biblioteca Motor usada para acesso ao mongoDB
- Biblioteca Pydantic para criação dos modelos e tipos
- Web socket para envio de notificações
- Biblioteca Jose para autenticação
- Comunicação entre as camadas com a classe Result
- Contratos de interfaces
- Tratamento de erros com classes personalizadas


\subsection[Implementação do módulo de processamento de dados]{Implementação do módulo de processamento de dados}
- Leitura dos dados de acordo com o processamento realizado anteriormente
- Uso da biblioteca pandas
- Funcionamento do BoxPlot para realizar analise estatística

\subsection[Implementação do frontend]{Implementação do frontend}
- Criação de paginas componentes de layouts
- Recharts
- Material UI
- Days JS
- Criação da camada de dados com o Context API
- Acesso externo a API
- Axios e fetch

\subsection[Adaptando a implementação para outros contextos]{Adaptando a implementação para outros contextos}
Discussão sobre a reutização do sistema para outros contextos....
- como fazer
- alterações necessarias