% \chapter{Abordagem/Análise/Modelação}\label{cap:metodology}

% Neste capítulo espera-se uma descrição detalhada do problema e da proposta de solução. 

% No caso de projetos de desenvolvimento de software, deverá deitar-se mão dos conceitos e ferramentas de Análise/Modelação estudadas durante o curso (por exemplo, diagramas UML ou outra linguagem gráfica). Deve-se indicar explicitamente as tarefas a desempenhar pelo sistema, e os atores que interagem com o mesmo. A descrição deve ter suficiente detalhes  para perceber as dificuldades associadas à resolução do problema.


\chapter{Methodology}\label{cap:metodology}

In this chapter it is expected a detailed description of the problem and proposed solution.

In the case of software development projects, there should include tools and concepts related to the modeling and analysis (such as UML diagrams or others). There should also describe the tasks that the system should implement and the authors that interact with it. The description should be detailed to understand the difficulties associated to the problem resolution.

\section[Tecnologias]{Tecnologias}
Texto ...

\section[Método de coleta e armazenamento de dados]{Método de coleta e armazenamento de dados}
- Recebimento dos dados pelo protocolo feito pelo outro professor
- protocolo esse que decodifico as informações e disponibilizo para uso

\section[Processo de desenvolvimento do software]{Processo de desenvolvimento do software}

- Definição do escopo - Comparar a ideia inicial com o escopo entregue (?)
- Organização das tarefas em um kanban
- Notion para documentação
- Github
- Reuniões semanais com o professor para discutir sobre o andamento


\section[Gestão das atividades do projeto]{Gestão das atividades do projeto}
- Historias de usuario
- Listagem do que deveria ser feito no kanban no notion

\section[Desafios para obtenção dos dados]{Desafios para obtenção dos dados}
Descrever mais sobre como foi complexo ter uma modelo de dados bem definido, como as empresas tem políticas que podem dificultar o acesso a informações que podem ser necessárias para o sucesso de um projeto como esse
