%\thispagestyle{empty}

Nesta dissertação de mestrado, o objetivo principal é desenvolver um sistema para monitoramento de sensores industriais, especificamente para a indústria de estampagem. O trabalho é motivado pela necessidade de sistemas de monitoramento em tempo real para rastrear a operação de máquinas na produção, possibilitando assim uma gestão baseada em dados. O sistema foi implementado utilizando Python no backend com FastAPI para a API, MongoDB para armazenamento de dados e NextJs para o painel de controle onde as informações são exibidas.

Os resultados indicam que o sistema é capaz de monitorar, analisar e apresentar dados dos sensores em tempo real, com um mecanismo de alerta que aciona notificações com base em parâmetros pré-definidos. O estado da arte foi revisado para entender melhor as técnicas e tecnologias emergentes em áreas relacionadas, como a Internet industrial das Coisas (IoT), big data e análise de dados em tempo real.

A implementação atual, embora mais simples em comparação com as soluções encontradas na literatura, serviu como uma prova de conceito válida e é altamente adaptável para futuras iterações com base no feedback do ambiente de produção real. Conclui-se que o sistema desenvolvido atende aos requisitos iniciais e oferece um certo grau de flexibilidade para se adaptar a outros contextos e fazer melhorias futuras.

A aplicação de \gls{AI} para análise de dados e manutenção preditiva de máquinas são direções prováveis para pesquisa e desenvolvimento de trabalhos futuros.

\mbox{}\linebreak
\noindent {\bf Palavras-chave:} Internet Industrial das Coisas (IIoT), Monitoramento em Tempo Real, Análise de Dados de Sensores, Data Lake.


%\vfill
%\pagebreak
%\mbox{}
%\vfill
%\pagebreak