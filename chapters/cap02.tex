% \chapter{Contexto e Tecnologias/Ferramentas}\label{cap:conceptual}

% Neste capítulo espera-se uma descrição genérica do problema e da área de intervenção: âmbito, conceitos e tecnologia e/ou uma revisão da literatura (``estado da arte''). No caso de um projeto eminentemente prático, devem ser descritas também as ferramentas usadas e a justificação para a sua escolha.

% Normalmente, este capítulo é dividido em múltiplas secções, de forma a compartimentar os tópicos abordados, facilitando assim a sua leitura e compreensão.

\chapter{Revisão literaria }\label{cap:conceptual}

In this chapter it is expected to have a generic description of the problem and area: scope, concepts and technology and/or a literature review (state-of-the-art). In case of a practical project, there should also be described the tools and the justification for their use.

Usually, this chapter is divided in multiple sections, to complement the topics.

trabalhos semelhantes já realizados, métodos de análise de dados, até discussões sobre visualização de dados e importância da gestão de dados no mercado atual


\section[Nome do artigo]{Nome do artigo}

Texto...