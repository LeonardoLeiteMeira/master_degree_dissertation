\chapter{State of the art}\label{cap:conceptual}

The literature review for the development of this project was carried out by researching projects that were similar to this one in some aspects, such as the receipt, storage, and processing of \gls{IoT} sensor data, and real-time information transmission.

\section{Data storage and big data}

In the context of data storage and big data, the importance of managing and analyzing large data sets has been emphasized in various studies. One of these studies is the work on the \gls{FQC}, a software designed to manage information from FASTQ files ~\cite{fqc2017}. Developed in Python and JavaScript, the \gls{FQC} aggregates data and generates metrics that are displayed on a dashboard. The software is capable of processing single-end or paired data, and can process batch files based on a specified directory.

It is important to note that the software allows customization, users can configure \gls{KPI}, charts, and other dashboard elements. The data processing layer of the \gls{FQC} bears similarities with industrial sensor monitoring systems of this dissertation, mainly in aspects of data access and processing. It operates by accessing a directory, processing the information, and making it available for later use, and also supports the execution of small batch functions.

The dashboard built by the FQC offers a variety of visualizations, including line charts, bar charts, and heat maps, among others. These visualizations are dynamically generated and can be configured using \gls{json} files \cite{mdnJson}, providing a flexible and user-friendly interface for data management and analysis.

Another contribution to the field of data storage and big data is the work \cite{ren2021data}, which focuses on predicting product quality through a data-driven approach. The \gls{KPI} are identified to serve as highly relevant state variables for product quality. Traditionally, these variables are measured through offline laboratory analyses, which introduces latency into the system.

The \gls{AI} component of the system is layered, where the lower layer deals with both categorized and uncategorized data to train and test the AI models. A Gaussian distribution is applied in the second layer to process the data, which are then fed into a semi-supervised training layer. The final layer provides the results of the predictions, thus closing the cycle.

Case studies presented in the article demonstrate the implementation of this system in industrial mineral processing. Although the method successfully addresses the issue of unmarked records, it requires a high degree of continuity in industrial systems, which is identified as a limitation.

The work of Ren et al. provides insights into the use of data for predictive quality control in industrial environments. The layered \gls{AI} model and the focus on \gls{KPI} are especially relevant for the future development of the system, which can use the analyses carried out over time as training data for the \gls{AI}.

Still within the scope of data storage and big data, Predictive Maintenance (PdM) becomes a very relevant strategy, both in the context of this project and in semiconductor manufacturing, as explained in \cite{susto2015machine}. The article \cite{susto2015machine} presents a multiple classifier approach to PdM, aiming to minimize downtime and associated costs. Three main categories for maintenance management are identified: Run to Failure, Preventive Maintenance, and Predictive Maintenance. The latter is emphasized for its ability to leverage historical data, forecasting algorithms, statistics, and engineering methods.

The paper uses several trained classification modules with different forecasting horizons to offer various performance trade-offs. Two main indicators are identified to reduce total operational costs: the frequency of unexpected breakdowns and the amount of unutilized lifespan. Linear regression is used as a statistical method for forecasting.

The approach is particularly relevant for systems that require real-time data analysis and efficient data storage, such as industrial sensor monitoring systems. It addresses the limitations associated with the lack of continuous data feed in industrial systems and offers a cost-based decision-making system for maintenance management.

Therefore, the article provides insights into the application of machine learning for predictive maintenance, especially in semiconductor manufacturing. The methodology can be particularly beneficial for industrial environments where minimizing downtime and operational costs are essential, and could be a logical evolution of this project, given the storage of historical data received in the system that can be used for predictive maintenance.

\section{Real-time Monitoring}

In the context of industrial sensor monitoring and real-time data analysis, the paper \cite{shafi2019precision} is relevant as it offers a comprehensive exploration of precision agriculture techniques, with a particular focus on IoT-based intelligent irrigation systems. These systems face similar challenges to those in industrial environments, such as latency, bandwidth limitations, and intermittent internet connectivity.

Edge computing (fog computing), as discussed in the article, emerges as a cutting-edge solution to these challenges. It aims to save energy and bandwidth, reducing failure rates and delays. This is particularly relevant for real-time data analysis and alert systems in industrial sensor monitoring. The Fog of Everything architecture, introduced in the article, offers a multi-layered approach that could be adapted for industrial applications aiming to improve service quality and efficiency in data storage. The methodologies and technologies discussed in the article provide ways of system organization for data storage, big data, and sensor monitoring systems.

The article ~\cite{REHMAN} explores the integration of \gls{BDA} with \gls{IIoT}, focusing on real-time data analysis, data management and storage, aspects that connect with the aim of the dissertation to develop a robust system for monitoring industrial sensors.

The article's discussion on real-time analysis can guide the development of the Data Reception Module, and the data processing module for historical data analysis in this project. In addition, the article's insights on data management and storage can offer paths to optimize the performance of the MongoDB database if necessary.

Although the article does not specifically discuss alert systems, its categorization of analysis techniques into descriptive, prescriptive, predictive, and preventive procedures can provide a framework for generating alerts based on predefined parameters. Moreover, the article's focus on interoperability and integration in IIoT systems may offer guidelines for the effective design and integration of the various modules in this project, such as the Database, Data Receiving Module, and \gls{API}.

An important project to highlight in real-time data monitoring is presented in the article ~\cite{Umakirthika2018}. This explores the use of Internet of Things (IoT) technologies to enhance vehicle safety. The article introduces an \gls{ODAS} System designed to identify obstacles on the road and alert drivers in real-time. The system uses embedded algorithms that detect obstacles based on various vehicle parameters, such as speed and steering angle. Once an obstacle is detected, its location is stored locally and sent to a cloud server periodically. The cloud server processes these data, confirms the presence of a real obstacle, and sends this information back to the vehicle's alert system, which provides audible and visual alerts to the driver.

This article's approach can provide important information for the design and implementation of real-time alert systems in an industrial environment, specifically, the use of IoT for data collection and cloud processing can be adapted to enhance the real-time analysis capabilities of this system. The article also discusses the challenges associated with implementing such a system, including data security and latency, which are important points to consider when transmitting real-time data and generating alerts about machine operation.

In the field of data storage and big data, machine learning algorithms have been widely studied for their ability to analyze and interpret large data sets. A review conducted in \cite{sarker2021machine}, elucidates various statistical and machine learning techniques pertinent to feature selection and data analysis. Methods such as Variance Analysis and Chi-Square tests are highlighted for their utility in identifying statistically significant features in data sets. These techniques are particularly relevant for systems that require real-time data analysis and decision-making, such as industrial sensor monitoring systems.

In addition, the article discusses the application of machine learning algorithms in various areas, potentially including industrial settings and the Internet of Things (IoT). Although the article does not specifically delve into real-time data analysis, the algorithms and methods presented can be adapted for such purposes. For instance, machine learning algorithms can be employed to predict sensor failures or other anomalies based on historical data, thereby enhancing the robustness and reliability of industrial monitoring systems.

In this way, the methodologies and algorithms discussed in \cite{sarker2021machine} provide guidance for the development of systems that require efficient data storage and real-time analysis capabilities.

\section{Data Processing and Analysis}
The article ~\cite{Lv2017} provides a comprehensive review of the current landscape of \gls{BDA}, covering various types of data, storage models, and analysis methods. The article also addresses the challenges and future research directions in \gls{BDA}, emphasizing the role of emerging technologies such as edge computing, machine learning, and blockchain.

In the context of this dissertation, the article's comprehensive treatment of data storage models and analysis methods is particularly relevant when it discusses storage models, including NoSQL databases like MongoDB, which can provide insights to optimize the database component of this project.

Furthermore, the exploration of various analysis methods by the article, such as machine learning algorithms and real-time analysis, can serve as a guide for future improvements in data analysis by the Data Processing Module in this project. The article also discusses the integration of edge computing for real-time analysis, which could be a future direction for this dissertation in order to make the system more scalable and efficient in an industrial environment.

Although the article covers a wide range of topics, its general principles and methodologies can be adapted to the industrial context of this dissertation. The article describes the practical applications of \gls{BDA}, and discusses challenges, such as data security and privacy, and future research directions.

Another relevant article is ~\cite{Zhang2018}, which provides a comprehensive review of the role of \gls{BDA} in the context of smart grids. The article explores various analysis techniques, including machine learning algorithms, statistical methods, and data mining, and their applications in smart grid systems. It also discusses the challenges and future directions in the field, such as data security, real-time analysis, and integration of renewable energy sources.

The exploration of machine learning algorithms and statistical methods by the article can be especially informative to enhance the real-time data analysis performed by the Data Processing Module in this project. Moreover, the article's discussion on real-time analysis is directly relevant to the focus of this dissertation, as it explores the challenges and solutions in implementing real-time analysis in smart grids, which could offer options for the development of this system's real-time analysis capabilities.

The paper also discusses data security challenges, which could be pertinent when considering the secure \gls{API} for managing access to sensor data in an industrial environment. Although the paper is focused on smart grids, its general principles and methodologies can be adapted to the context of industrial sensor monitoring of this dissertation.

An important work that discusses data structuring is presented in the paper ~\cite{Fan2021}. This paper presents a critical review of data-based methods for building energy modeling and their practical applications for improving building performance. Although the main focus of the paper is on building energy systems, its methodological approach and findings provide important insights for this project. The paper categorizes analysis techniques into descriptive, prescriptive, predictive, and preventive procedures, and discusses the transition from traditional data-based methods to big data-based approaches.

The detailed discussion of the article on data-based methods can guide the statistical data analysis carried out by the Data Processing Module. Moreover, the exploration of big data-based approaches by the article may offer new perspectives to optimize the MongoDB database used in this system. The article discusses the challenges and opportunities in transitioning from traditional data storage and analysis methods to big data-based approaches, which may be relevant to enhance the system's adaptability and performance in an industrial environment.

Although the article focuses on building performance, its general principles and methodologies on data analysis can be adapted to the context of industrial sensor monitoring of this dissertation. The article also goes through the practical applications and challenges of data-based methods, which can serve as a guide for future improvements and adaptations of the system based on feedback from the real production environment.


\section{Conclusion}
In the review of the state of the art, various approaches were analyzed, covering areas such as data storage and big data, real-time monitoring, data analysis processes, and methodology for software products. These topics are crucial for understanding and developing a robust system for monitoring industrial sensors. The reviewed academic articles provided important information in both technical aspects and broader context considerations. From a technical standpoint, the methods and technologies explored in the consulted works helped to understand issues related to efficiency in real-time communication between sensors and servers, as well as the structuring of databases to accommodate and retrieve large volumes of information.

Furthermore, the state of the art provided guidelines on advancements in the application of \gls{AI} for data analysis. These applications prove to be especially relevant for areas such as predictive maintenance, which benefit from real-time data analysis, and from the ability to make reliable predictions about future system states. This integration of AI can pave the way for more proactive and less reactive monitoring, contributing to the overall increase in efficiency and reduction of operational costs.

Regarding the context, the review of the state of the art also helped to identify the challenges and opportunities that may shape future versions of this monitoring system. Trends in emerging technologies and industrial practices can be informed by these academic reviews, allowing the project to stay aligned with the most recent advancements in the field and prepared to meet new demands that may arise.
Regarding the current implementation, it is pertinent to mention that it presents a lower level of complexity compared to the solutions detailed in the discussed literature since the goal is to develop a first functional version of the system, which can be iteratively improved. This approach allows for faster validation in real production environments, paving the way for refinements based on practical feedback and observed performance.