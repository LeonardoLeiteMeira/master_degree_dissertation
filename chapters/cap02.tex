\chapter{Estado da arte}\label{cap:conceptual}
%TODO sigla IOT
A revisão literária para o desenvolvimento desse projeto se deu pela pesquisa de projetos que fosse semelhante a esse em algum aspectos, como o recebimento, armazenada e processamento de dados de sensores IoT, e transmissão de informação em tempo real.

\section{Armazenamento de dados e big data}

%Sigla FQC - FastQ Quality Control
No contexto do armazenamento de dados e big data a importância de gerenciar e analisar conjuntos de dados grandes tem sido enfatizada em diversos estudos. Um desses estudos é o trabalho sobre o FQC, um software projetado para gerenciar informações de arquivos FASTQ ~\cite{fqc2017}. Desenvolvido em Python e JavaScript, o FQC agrega dados e gera métricas que são exibidas em um painel. O software é capaz de processar dados de extremidade única ou emparelhados, e pode processar arquivos em lote com base em um diretório especificado.

É importante notar que o software permite personalização, os usuários podem configurar Indicadores Chave de Desempenho (KPIs), gráficos e outros elementos do painel. A camada de processamento de dados do FQC apresenta semelhanças com sistemas de monitoramento de sensores industriais dessa dissertação, principalmente nos aspectos de acesso e processamento de dados. Ele opera acessando um diretório, processando as informações e tornando-as disponíveis para uso posterior, e também suporta a execução de pequenas funções em lotes.

%TODO Sigla JSON
O painel construído pelo FQC oferece uma variedade de visualizações, incluindo gráficos de linhas, gráficos de barras e mapas de calor, entre outros. Essas visualizações são geradas dinamicamente e podem ser configuradas usando arquivos JSON, proporcionando uma interface flexível e amigável para o gerenciamento e análise de dados.

%TODO Sigla KPI
Outra contribuição para o campo do armazenamento de dados e big data é o trabalho de Ren et al. \cite{ren2021data}, \texttt{Quality Mining in a Continuous Production Line based on an Improved Genetic Algorithm Fuzzy Support Vector Machine}, que se concentra em prever a qualidade do produto por meio de uma abordagem orientada por dados. Indicadores Chave de Desempenho (KPIs) são identificados para servir como variáveis de estado altamente relevantes para a qualidade do produto. Tradicionalmente, essas variáveis são medidas por meio de análises laboratoriais offline, o que introduz latência no sistema.

%TODO SIGLA IA
O componente de inteligência artificial (IA) do sistema é estruturado em camadas, em que a camada inferior lida tanto com dados categorizados quanto não categorizados para treinar e testar os modelos de IA. Uma distribuição Gaussiana é aplicada na segunda camada para processar os dados, que são então alimentados para uma camada de treinamento semi-supervisionado. A camada final disponibiliza os resultados das previsões, fechando assim o ciclo.

Estudos de caso apresentados no artigo demonstram a implementação desse sistema em processos industriais de processamento de minerais. Embora o método aborde com sucesso a questão de registros sem marcação, ele exige um alto grau de continuidade em sistemas industriais, o que é identificado como uma limitação.

O trabalho de Ren et al. fornece ideias sobre o uso de dados para controle preditivo de qualidade em ambientes industriais. O modelo de IA em camadas e o foco em KPIs são especialmente relevantes para o desenvolvimento futuro do sistema, que pode utilizar as analises realizadas ao longo do tempo para usar de dados de treinamento para a IA.

%TODO sigla pdm
Ainda no âmbito do armazenamento de dados e big data, a manutenção preditiva (PdM) se torna uma estratégia muito relevante, tanto no contexto desse projeto quanto na fabricação de semicondutores \cite{susto2015machine}, como explicado em \texttt{Machine Learning for Predictive Maintenance: A Multiple Classifier Approach}. O artigo de Susto et al.\cite{susto2015machine} apresenta uma abordagem de múltiplos classificadores para a PdM, com o objetivo de minimizar o tempo de inatividade e os custos associados. Três categorias principais para o gerenciamento de manutenção são identificadas: Executar até Falhar, Manutenção Preventiva e Manutenção Preditiva. Esta última é enfatizada por sua capacidade de aproveitar dados históricos, algoritmos de previsão, estatísticas e métodos de engenharia.

O artigo utiliza vários módulos de classificação treinados com horizontes de previsão diferentes para oferecer várias compensações de desempenho. Dois principais indicadores são identificados para reduzir os custos operacionais totais: a frequência de quebras inesperadas e a quantidade de vida útil não aproveitada. A regressão linear é utilizada como método estatístico para previsão.

A abordagem é particularmente relevante para sistemas que requerem análise de dados em tempo real e armazenamento eficiente de dados, como sistemas de monitoramento de sensores industriais. Ela aborda as limitações associadas à falta de alimentação contínua de dados em sistemas industriais e oferece um sistema de tomada de decisão com base em custos para o gerenciamento de manutenção.

Portanto, o trabalho de Susto et al. oferece insights sobre a aplicação de aprendizado de máquina para a manutenção preditiva, especialmente na fabricação de semicondutores. A metodologia pode ser particularmente benéfica para ambientes industriais onde minimizar o tempo de inatividade e os custos operacionais são essenciais, podendo ser uma evolução lógica desse projeto, dado o armazenamento dos dados históricos recebidos no sistema que podem ser usados para a realização de manutenção preditiva.


\section{Monitoramento em tempo real}

No contexto do monitoramento de sensores industriais e análise de dados em tempo real, o artigo de Shafi et al. \cite{shafi2019precision} \texttt{Precision agriculture techniques and practices: From considerations to applications} possui relevância pois oferece uma exploração abrangente das técnicas de agricultura de precisão, com foco particular em sistemas inteligentes de irrigação baseados em IoT. Esses sistemas enfrentam desafios semelhantes aos de ambientes industriais, como latência, limitações de largura de banda e conectividade intermitente com a Internet.
 
A computação de borda (fog computing), como discutida no artigo, surge como uma solução de ponta para esses desafios. Ela visa economizar energia e largura de banda, reduzindo as taxas de falha e atrasos. Isso é particularmente relevante para a análise de dados em tempo real e sistemas de alerta no monitoramento de sensores industriais. A arquitetura Fog of Everything, introduzida no artigo, oferece uma abordagem em várias camadas que poderia ser adaptada para aplicações industriais visando melhorar a qualidade do serviço e a eficiência no armazenamento de dados. As metodologias e tecnologias discutidas no artigo fornecem formas de organização do sistema para o armazenamento de dados, big data e sistemas de monitoramento de sensores.

%TODO Sigla BDA e IIoT
Já o artigo \texttt{The Role of Big Data Analytics in Industrial Internet of Things} de Rehman et al.~\cite{REHMAN} explora a integração da Análise de Grandes Dados (BDA) com a Internet Industrial das Coisas (IIoT), com foco em análise de dados em tempo real, gerenciamento e armazenamento de dados, aspectos se conectam com o objetivo da dissertação de desenvolver um sistema robusto para o monitoramento de sensores industriais.

A discussão do artigo sobre análises em tempo real pode orientar o desenvolvimento do Módulo de Recebimento de dados, e o modulo de processamento de dados para análise de dados históricos neste projeto. Além disso, as percepções do artigo sobre gerenciamento e armazenamento de dados podem oferecer caminhos para otimizar o desempenho do banco de dados MongoDB caso necessário.

Embora o artigo não discuta especificamente sistemas de alerta, sua categorização de técnicas de análise em procedimentos descritivos, prescritivos, preditivos e preventivos pode fornecer um quadro para gerar alertas com base em parâmetros predefinidos. Além disso, o foco do artigo na interoperabilidade e integração em sistemas IIoT pode oferecer diretrizes para o design e integração eficazes dos diversos módulos neste projeto, como o Banco de Dados, Módulo de Recebimento de Dados e API.

%TODO sigla odas
Um projeto importante a ser destacado no monitoramento de dados em tempo real é apresentado no artigo \texttt{Internet of Things in Vehicle Safety – Obstacle Detection and Alert System} de Umakirthika et al.~\cite{Umakirthika2018}. Este explora de maneira abrangente o uso de tecnologias de Internet das Coisas (IoT) para aprimorar a segurança veicular. O artigo introduz um Sistema de Detecção de Obstáculos e Alerta (ODAS) projetado para identificar obstáculos na estrada e alertar os motoristas em tempo real. O sistema utiliza algoritmos incorporados que detectam obstáculos com base em vários parâmetros do veículo, como velocidade e ângulo de direção. Uma vez que um obstáculo é detectado, sua localização é armazenada localmente e enviada para um servidor em nuvem periodicamente. O servidor em nuvem processa esses dados, confirma a presença de um obstáculo real e envia essa informação de volta para o sistema de alerta do veículo, que fornece alertas audíveis e visuais ao motorista.

Essa abordagem do artigo pode oferecer informações importantes para o projeto e implementação do sistemas de alerta em tempo real em um ambiente industrial, especificamente, o uso de IoT para coleta de dados e processamento em nuvem pode ser adaptado para aprimorar as capacidades de análise em tempo real deste sistema. O artigo também discute os desafios associados à implementação de um sistema desse tipo, incluindo segurança de dados e latência, que são pontos importantes a se considerar quando se transmite dados em tempo real e gera alertas sobre o funcionamento das maquinas.

No campo do armazenamento de dados e big data, algoritmos de aprendizado de máquina têm sido amplamente estudados por sua capacidade de analisar e interpretar grandes conjuntos de dados. Uma revisão realizada por Sarker \cite{sarker2021machine}, em \texttt{Machine Learning: Algorithms, Real-World Applications and Research Directions}, esclarece várias técnicas estatísticas e de aprendizado de máquina pertinentes para seleção de características e análise de dados. Métodos como Análise de Variância e testes Qui-Quadrado são destacados por sua utilidade na identificação de características estatisticamente significativas nos conjuntos de dados. Essas técnicas são particularmente relevantes para sistemas que requerem análise de dados em tempo real e tomada de decisões, como sistemas de monitoramento de sensores industriais.

Além disso, o artigo discute a aplicação de algoritmos de aprendizado de máquina em várias áreas, potencialmente incluindo configurações industriais e a Internet das Coisas (IoT). Embora o artigo não se aprofunde especificamente na análise de dados em tempo real, os algoritmos e métodos apresentados podem ser adaptados para tais fins. Por exemplo, algoritmos de aprendizado de máquina podem ser empregados para prever falhas de sensores ou outras anomalias com base em dados históricos, aumentando assim a robustez e confiabilidade dos sistemas de monitoramento industrial.

Dessa forma, as metodologias e algoritmos discutidos por Sarker \cite{sarker2021machine} oferecem direcionamentos para o desenvolvimento de sistemas que exigem armazenamento eficiente de dados e capacidades de análise em tempo real.


\section{Processamento e análise de dados}
%Sigla DBA
O artigo \texttt{Next-Generation Big Data Analytics: State of the Art, Challenges, and Future Research Topics} de Lv et al.~\cite{Lv2017} oferece uma revisão abrangente do cenário atual da Análise de Grandes Dados (BDA), abrangendo diversos tipos de dados, modelos de armazenamento e métodos de análise. O artigo também aborda os desafios e direções futuras de pesquisa em BDA, enfatizando o papel de tecnologias emergentes como computação de borda, aprendizado de máquina e blockchain.

No contexto desta dissertação, que visa desenvolver um sistema robusto para o monitoramento de sensores industriais, o tratamento abrangente do artigo sobre modelos de armazenamento de dados e métodos de análise é relevante principalmente quando discute sobre modelos de armazenamento, incluindo bancos de dados NoSQL como o MongoDB, pode oferecer insights para otimizar o componente do banco de dados deste projeto.

Além disso, a exploração de diversos métodos de análise pelo artigo, como algoritmos de aprendizado de máquina e análise em tempo real, pode servir como guia para, no futuro, aprimorar a análise de dados pelo Módulo de Processamento de Dados neste projeto. O artigo também discute a integração da computação de borda para análise em tempo real, o que poderia ser uma direção futura para esta dissertação a fim de tornar o sistema mais escalável e eficiente em um ambiente industrial.

Embora o artigo cubra uma ampla gama de tópicos, seus princípios e metodologias gerais podem ser adaptados ao contexto industrial desta dissertação. O artigo não apenas descreve sobre as aplicações práticas da BDA, mas também discute os desafios, como segurança de dados e privacidade, e as direções futuras de pesquisa.


Outro artigo relevante é \texttt{Big data analytics in smart grids: a review} de Zhang et al.~\cite{Zhang2018}, que oferece uma revisão abrangente do papel da Análise de Grandes Dados (BDA) no contexto das redes inteligentes. O artigo explora diversas técnicas de análise, incluindo algoritmos de aprendizado de máquina, métodos estatísticos e mineração de dados, e suas aplicações em sistemas de redes inteligentes. Ele também discute os desafios e direções futuras no campo, como segurança de dados, análise em tempo real e integração de fontes de energia renovável.

A exploração de algoritmos de aprendizado de máquina e métodos estatísticos pelo artigo pode ser especialmente informativa para aprimorar a análise de dados em tempo real realizada pelo Módulo de Processamento de Dados neste projeto. Além disso, a discussão do artigo sobre análise em tempo real é diretamente relevante para o foco desta dissertação, pois ele explora os desafios e soluções na implementação de análise em tempo real em redes inteligentes, o que poderia oferecer opções para o desenvolvimento das capacidades de análise em tempo real deste sistema.

O artigo também discute desafios de segurança de dados, que poderiam ser pertinentes ao considerar a API segura para gerenciar o acesso aos dados dos sensores em um ambiente industrial. Embora o artigo seja centrado em redes inteligentes, seus princípios e metodologias gerais podem ser adaptados ao contexto de monitoramento de sensores industriais desta dissertação.

Um trabalho importante que mostra sobre a estruturação dos dados é apresentado no artigo \texttt{Advanced data analytics for enhancing building performances: From data-driven to big data-driven approaches} de Fan et al.~\cite{Fan2021}. Esse artigo apresenta uma revisão crítica dos métodos baseados em dados para modelagem de energia de edifícios e suas aplicações práticas para a melhoria do desempenho de edifícios. Embora o foco principal do artigo seja em sistemas de energia de edifícios, sua abordagem metodológica e descobertas oferecem importantes informações para este projeto. O artigo categoriza técnicas de análise em procedimentos descritivos, prescritivos, preditivos e preventivos, e discute a transição de métodos tradicionais baseados em dados para abordagens baseadas em big data.

A discussão detalhada do artigo sobre métodos baseados em dados pode orientar a análise estatística de dados realizada pelo Módulo de Processamento de Dados. Além disso, a exploração de abordagens baseadas em big data pelo artigo pode oferecer novas perspectivas para otimizar o banco de dados MongoDB utilizado neste sistema. O artigo discute os desafios e oportunidades na transição de métodos tradicionais de armazenamento e análise de dados para abordagens baseadas em big data, o que pode ser relevante para aprimorar a adaptabilidade e o desempenho do sistema em um ambiente industrial.

Embora o artigo foque em desempenho de edifícios, seus princípios e metodologias gerais sobre análise de dados podem ser adaptados para o contexto de monitoramento de sensores industriais desta dissertação. O artigo também passa pelas aplicações práticas e desafios dos métodos baseados em dados, o que pode servir como guia para futuros aprimoramentos e adaptações do sistema com base no feedback do ambiente de produção real.

\section{Metodologia para desenvolvimento}
No campo das metodologias de desenvolvimento de software, o artigo de Radha Shankarmani intitulado "\texttt{Agile Methodology Adoption: Benefits and Constraints}" \cite{shankarmani2012agile} oferece informações valiosas sobre a natureza iterativa dos processos ágeis. O artigo enfatiza a importância dos ciclos iterativos que entregam resultados concretos e utilizáveis após cada iteração. Além disso, ele discute o papel dos pontos de verificação do cliente para revisão de qualidade ao final de cada iteração, garantindo que o trabalho esteja alinhado com os padrões de qualidade pré-definidos. O artigo também destaca a importância das revisões pós-versão que geram feedback para planos de melhoria, o que é de extrema importância para o processo.

Essas práticas ágeis são particularmente relevantes para o atual projeto. Dado o foco na análise de dados em tempo real e sistemas de alerta, a abordagem iterativa e os mecanismos de feedback discutidos no artigo de Shankarmani podem ser instrumentais, especificamente, os ciclos iterativos podem facilitar o desenvolvimento incremental e o aperfeiçoamento dos módulos do sistema para atender melhor as necessidades de acordo com os feedbacks recebidos. Além disso, o feedback gerado pelas revisões pós-versão pode ser inestimável para melhorar a qualidade e o desempenho gerais do sistema, especialmente quando o software é colocado em produção.

Assim, as metodologias ágeis discutidas no artigo podem servir como um quadro orientador para gerenciar o processo evolutivo do sistema ao longo do tempo, garantindo sua adaptabilidade e capacidade de resposta adequado ao contexto e necessidades da empresa.


\section{Conclusão}
Na revisão do estado da arte, diversas abordagens foram analisadas, abrangendo áreas como armazenamento de dados e big data, monitoramento em tempo real, processos de análise de dados e metodologia para produtos de software. Estes tópicos são cruciais para o entendimento e desenvolvimento de um sistema robusto de monitoramento de sensores industriais. Os artigos acadêmicos revisados forneceram informações importantes tanto em aspectos técnicos quanto em considerações de contexto mais amplo. De um ponto de vista técnico, os métodos e tecnologias explorados nos trabalhos consultados ajudaram a elucidar questões relacionadas à eficiência na comunicação em tempo real entre sensores e servidores, bem como à estruturação de bancos de dados para acomodar e recuperar grandes volumes de informação.

Mais ainda, o estado da arte ofereceu diretrizes sobre os avanços na aplicação de Inteligência Artificial para a análise de dados. Estas aplicações mostram-se especialmente relevantes para áreas como manutenção preditiva, que não apenas beneficiam da análise de dados em tempo real, mas também da capacidade de fazer previsões confiáveis sobre futuros estados de sistema. Essa integração de IA pode abrir portas para um monitoramento mais proativo e menos reativo, contribuindo para o aumento geral da eficiência e redução de custos operacionais.

No que tange ao contexto, a revisão do estado da arte também ajudou a identificar os desafios e oportunidades que podem moldar futuras versões deste sistema de monitoramento. As tendências em tecnologias emergentes e práticas industriais podem ser informadas por essas revisões acadêmicas, permitindo que o projeto se mantenha alinhado com os avanços mais recentes no campo e preparado para atender a novas exigências que possam surgir.

Em relação à implementação atual, é pertinente mencionar que ela apresenta um nível de complexidade menor quando comparada às soluções detalhadas na literatura discutida pois está em linha com o objetivo de desenvolver uma primeira versão funcional do sistema, que pode ser iterativamente aprimorada. Este enfoque permite uma validação mais rápida em ambientes de produção real, abrindo caminho para refinamentos com base em feedback prático e desempenho observado.

