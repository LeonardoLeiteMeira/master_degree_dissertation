%\thispagestyle{empty}

In this master's dissertation, the main objective is to develop a system for monitoring industrial sensors, specifically for the stamping industry. The work is motivated by the need for real-time monitoring systems to track machine operation in production, thus enabling data-driven management. The system was implemented using Python on the backend with FastAPI for the API, MongoDB for data storage, and NextJs for the dashboard where the information is displayed.

The results indicate that the system is capable of monitoring, analyzing, and presenting sensor data in real-time, with an alert mechanism that triggers notifications based on predefined parameters. The state of the art was reviewed to better understand emerging techniques and technologies in related areas, such as industrial Internet of Things (IoT), big data, and real-time data analysis.

The current implementation, although simpler compared to the solutions found in the literature, served as a valid proof of concept and is highly adaptable for future iterations based on feedback from the real production environment. It is concluded that the developed system meets the initial requirements and offers a certain degree of flexibility to adapt to other contexts and make future enhancements.

The application of \gls{AI} for data analysis and predictive maintenance of machines are likely directions for research and development of future work.

\mbox{}\linebreak
\noindent {\bf Keywords: Industrial Internet of Things (IIoT), Real-time Monitoring, Sensor Data Analytics, Data Lake} 

%\vfill
%\pagebreak
%\mbox{}
%\vfill
%%\pagebreak