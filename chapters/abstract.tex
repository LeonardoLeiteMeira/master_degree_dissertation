%\thispagestyle{empty}

Nesta dissertação de mestrado, o principal objetivo é desenvolver um sistema robusto e eficiente para o monitoramento de sensores industriais, mais especificamente para a industria de estampagem. O trabalho é motivado pela necessidade de sistemas de monitoramento em tempo real para acompanhar o funcionamento das maquinas na produção, e assim ter uma gestão mais orientada a dados. O sistema foi implementado usando Python no backend com FastAPI na API, MongoDB para armazenamento de dados, e NextJs para o dashboard onde as informações são exibidas.

Os resultados indicam que o sistema é capaz de monitorar, analisar e apresentar dados dos sensores em tempo real, com um mecanismo de alerta que aciona notificações baseadas em parâmetros pré-definidos. O estado da arte foi revisado para compreender melhor as técnicas e tecnologias emergentes em áreas relacionadas, como Internet das Coisas (IoT) industrial, big data e análise de dados em tempo real.

A implementação atual, embora mais simples em comparação com as soluções encontradas na literatura, serviu como uma prova de conceito válida e é altamente adaptável para futuras iterações baseadas no feedback do ambiente de produção real. Conclui-se que o sistema desenvolvido atende aos requisitos iniciais e oferece um certo grau de flexibilidade para adaptar em outros contextos e realizar futuros aprimoramentos. 

A aplicação de Inteligência Artificial para análise de dados e manutenção preditiva das máquinas são prováveis direções para pesquisa e desenvolvimento de trabalhos futuros. 

\mbox{}\linebreak
\noindent {\bf Keywords: Industrial Internet of Things (IIoT), Real-time Monitoring, Sensor Data Analytics, Data Lake} 

%\vfill
%\pagebreak
%\mbox{}
%\vfill
%%\pagebreak