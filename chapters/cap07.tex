\chapter{Resultados e Avaliação}\label{cap:results}
%TODO 21ref para um artigo com vantagens de usar um sistema de monitoramento em tempo real
Neste capítulo, será apresentada uma avaliação do sistema desenvolvido, focando em seus benefícios e vantagens competitivas, bem como os resultados de testes e validações realizadas. O sistema pode ser modificado de tal forma que pode ser adaptado para outros contextos, onde informações geradas por sensores necessitam ser visualizadas em tempo real, e onde o processamento e análise de dados históricos são requeridos.

A implementação deste sistema pode auxiliar na identificação mais rápida de problemas na produção, graças ao monitoramento em tempo real. Além disso, os insights gerados pela análise histórica de dados podem ser instrumentalizados para otimizar processos de produção. 

%TODO 22ref para artigo diferencia competitiva para gestão orientada a dados
A adoção de um sistema como este representa uma vantagem competitiva, potencialmente elevando a eficiência da empresa em relação à concorrência. No entanto, é importante observar que o software ainda não foi implementado em um ambiente de produção, e portanto, existe um conjunto ainda não mapeado de melhorias que devem ser realizadas para que o sistema opere da forma mais eficaz possível.


\section{Benefícios}\label{sec:benfits}

Uma série de benefícios é oferecida pelo sistema desenvolvido. Em primeiro lugar, o monitoramento abrangente das máquinas em toda a planta industrial é facilitado. Por meio deste monitoramento, informações cruciais sobre o estado operacional de cada máquina são exibidas para os funcionários em tempo real.

Adicionalmente, o sistema proporciona a capacidade de tomar ações mais rápidas em caso de problemas operacionais, e como resultado, o tempo improdutivo das máquinas pode ser reduzido, mitigando prejuízos associados à produção.

%TODO 23ref para artigo de gestão orientada a dados
Outro aspecto vantajoso reside na avaliação de eficácia das medidas de manutenção implementadas. Por meio de acompanhamento histórico, o sistema permite visualizar os efeitos de ações tomadas para a manutenção das máquinas, deste modo, decisões mais informadas e efetivas podem ser tomadas rapidamente, ampliando os benefícios para outras máquinas na planta.

Finalmente, os dados históricos do sistema contribuem também para a geração de insights a partir da sua análise. Esta análise pode fornecer uma nova perspectiva para monitorar o desempenho das máquinas, e anomalias operacionais podem ser identificadas e medidas corretivas específicas podem ser implementadas para melhorar determinados indicadores.

\section{Vantagens competitivas}\label{sec:competitive}
O sistema desenvolvido oferece um conjunto de vantagens competitivas que se manifestam principalmente quando contrastadas com empresas que não implementam uma solução similar. Inicialmente, a falta de um sistema de monitoramento adequado pode resultar em operações abaixo do potencial eficiente para uma empresa. Este cenário gera custos adicionais em manutenção, perda de equipamentos e até mesmo desperdício de materia prima.

%TODO 24ref para artigo de gestão orientada a dados - redução de custos
Por outro lado, a implementação de um sistema de monitoramento robusto, proporciona benefícios à eficiência da produção, manutenção e desenvolvimento de produtos e serviços. O monitoramento em tempo real e a análise de dados históricos permitem a otimização de várias operações, desde a identificação rápida de problemas até a implementação de ações corretivas e preventivas. 

%TODO 25ref para artigo de gestão orientada a dados - melhoria da operação, preço e qualidade do serviço
Deste modo, a qualidade dos produtos ou serviços é significativamente melhorada. Além disso, as informações geradas podem ser utilizadas para uma produção mais enxuta e eficiente, o que reduz os custos gerais e, por consequência, pode tornar o produto ou serviço mais competitivo em termos de preço.

Portanto, as vantagens competitivas geradas pelo uso deste sistema são multifacetadas, englobando não apenas a eficiência operacional, mas também a qualidade e o custo de produtos e serviços. Essas melhorias conjuntas possibilitam que a empresa adquira uma posição mais sólida e vantajosa no mercado em que atua.

\section{Realização de testes e validações}\label{sec:tests}

%TODO 26Ref para artigo com a importancia de testar o software em ambiente produtivo
Embora todos os requisitos levantados para o sistema tenham sido atendidos e as histórias de usuários detalhadas tenham sido completadas, uma avaliação significativa do sistema em um ambiente de produção ainda não foi realizada. A importância de testar e validar um sistema de software em um ambiente real não pode ser subestimada, pois a realização de testes é fundamental para avaliar a adequação do sistema às necessidades práticas, enquanto a validação assegura que o sistema cumpre os requisitos estabelecidos. 

%TODO 27ref para artigo sobre as etapas ideias de desenvolvimento de software
%TODO 28ref para artigo sobre influencia de um sistema top-down ou cascata para testar
O desenvolvimento de sistemas de software é um processo iterativo que envolve uma série de etapas, incluindo requisitos, design, implementação, testes e manutenção. Cada uma dessas etapas requer revisão e ajustes com base nos resultados dos testes e validações. A falta de uma etapa rigorosa de testes e validações pode resultar em várias deficiências, tanto técnicas quanto funcionais, que podem não apenas afetar o desempenho do sistema, mas também torná-lo impraticável para uso em um ambiente de produção.

Sem uma série adequada de testes, o sistema é susceptível a falhas que podem ser tanto técnicas, como relacionadas aos requisitos funcionais. Estas falhas podem ser pequenas, mas têm o potencial de escalonar e comprometer a integridade do sistema. Assim, a ausência de testes e validações em um ambiente real representa uma lacuna importante que deve ser abordada para assegurar a robustez e eficácia do sistema.
