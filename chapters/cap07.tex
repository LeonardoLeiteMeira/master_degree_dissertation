\chapter{Results and Evaluation}\label{cap:results}

In this chapter, an evaluation of the developed system will be presented, focusing on its benefits and competitive advantages, as well as the results of tests and validations carried out. The system can be modified in such a way that it can be adapted to other contexts, where information generated by sensors needs to be visualized in real time, and where the processing and analysis of historical data are required.

The implementation of this system can assist in the quicker identification of production problems, thanks to real-time monitoring. In addition, the insights generated by the historical data analysis can be instrumentalized to optimize production processes \cite{raczSzabo2020realTime}.

The adoption of a system like this represents a competitive advantage, potentially raising the company's efficiency in relation to the competition \cite{ng2011realTime}. However, it is important to note that the software has not yet been implemented in a production environment, and therefore, there is an unmapped set of improvements that must be made for the system to operate as effectively as possible.

\section{Benefits}\label{sec:benfits}

A series of benefits is offered by the developed system. Firstly, comprehensive monitoring of machines throughout the industrial plant is facilitated. Through this monitoring, crucial information about the operational status of each machine is displayed to employees in real time. 

Additionally, the system provides the ability to take quicker actions in case of operational issues, and as a result, the downtime of machines can be reduced, mitigating losses associated with production. Consequently, it allows for timely oversight and understanding of the machinery's current state. The use of such a system not only ensures that the machinery is functioning optimally, but it also aids in the early detection of potential issues, making it an essential tool in modern industrial operations.

Another advantageous aspect lies in the evaluation of the effectiveness of the implemented maintenance measures. Through historical tracking, the system allows for the visualization of the effects of actions taken for machine maintenance, thus, more informed and effective decisions can be made quickly, extending the benefits to other machines in the plant.

The developed system features a modular characteristic, which allows the integration of new modules to expand its functionalities. A practical example of this is the possibility of adding a module intended for working with data from computer vision. This module would be capable of evaluating and monitoring the quality of production in real time, identifying defects with greater precision and efficacy. Additionally, another module that can be integrated refers to the access to historical data from the machines, providing a retrospective view and facilitating access to this information. Finally, considering the importance of maintenance, a module for predictive maintenance can be implemented. This module, by using the received data, could develop a model that understands the operation pattern of the machines, allowing more assertive and optimized interventions.

Finally, the system's historical data also contributes to the generation of insights from its analysis. This analysis can provide a new perspective for monitoring machine performance, and operational anomalies can be identified and specific corrective measures can be implemented to improve certain indicators.

\section{Competitive Advantages}\label{sec:competitive}
The developed system offers a set of competitive advantages that are mainly manifested when contrasted with companies that do not implement a similar solution. Initially, the lack of an adequate monitoring system can result in operations below the efficient potential for a company. This scenario generates additional costs in maintenance, equipment loss, and even waste of raw materials.

On the other hand, the implementation of a robust monitoring system provides benefits to the efficiency of production, maintenance, and development of products and services. Real-time monitoring and analysis of historical data allow the optimization of various operations, from the quick identification of problems to the implementation of corrective and preventive actions.

Thus, the quality of products or services is significantly improved, as the information provides data-driven management, thereby facilitating a leaner and more efficient production, which reduces overall costs and, consequently, can make the product or service more competitive in terms of price \cite{glowalla2014processDriven}.

Therefore, the competitive advantages generated by the use of this system are multifaceted, encompassing operational efficiency, and the quality and cost of products and services. These joint improvements enable the company to acquire a more solid and advantageous position in the market in which it operates.

\section{Conducting tests and validations}\label{sec:tests}
Although all the requirements raised for the system have been met and the detailed user stories have been completed, a significant evaluation of the system in a production environment has not yet been carried out. The importance of testing and validating a software system in a real environment cannot be underestimated \cite{leTraon1999selfTestable}, as conducting tests is fundamental to assess the system's suitability to practical needs, while validation ensures that the system meets the established requirements.

The development of software systems is an iterative process that involves a series of steps, including requirements, design, implementation, testing, and maintenance. Each of these steps requires review and adjustments based on the results of tests and validations \cite{coleman2006softwareProcess}. The lack of a rigorous testing and validation stage can result in various deficiencies, both technical and functional, which can affect the system's performance and render it impractical for use in a production environment.

Without a proper series of tests, the system is susceptible to failures that can be both technical and related to functional requirements. These failures may be minor, but they have the potential to escalate and compromise the system's integrity. Thus, the absence of tests and validations in a real environment represents a significant gap that must be addressed to ensure the system's robustness and effectiveness.