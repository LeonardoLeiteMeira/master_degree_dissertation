\section[Implementação do banco de dados]{Implementação do banco de dados}


Dentro da implementação do sistema, o MongoDB foi usado para armazenar todas as informações do sistema. Este banco de dados, orientado a documentos, permitiu uma organização flexível dos dados, facilitando o armazenamento de diferentes dados que podem ser recebidos pelo modulo de recebimento de dados, e facilitando a criação de camadas de processamento. A estruturação dos bancos de dados e suas respectivas coleções foi pensada para facilitar tanto a inserção quanto a consulta de informações.

Em relação à organização dos dados, os seguintes bancos de dados foram criados:

\begin{itemize}
    \item \textbf{Users}: Armazena informações referentes aos usuários. Possui coleções que registram tentativas de login, detalhes pessoais dos usuários e tokens associados a eles.
    
    \item \textbf{Notification}: Destinado às notificações do sistema. Atualmente, este banco contém apenas notificações associadas aos alertas das máquinas, gerados pelos dados recebidos dos sensores junto com os parâmetros armazenados.
    
    \item \textbf{Downtime}: Armazena duas coleções, uma com os dados lidos das planilhas de parada das máquinas, e outro com esses dados tratados. Esse banco de dados com essas coleções são apenas para simular como ficaria os dados de parada das maquinas, caso eles fosses inseridos no sistema.
    
    \item \textbf{Raw Data}: Este banco é dedicado ao armazenamento de dados brutos oriundos de diferentes sensores. Cada tipo de sensor, como os sensores de pressão, tem sua própria coleção, garantindo um agrupamento das informações que facilita a análise.
    
    \item \textbf{Processed Data}: Como o próprio nome sugere, armazena dados que já passaram por uma etapa de processamento. Assim, dados interpretados de diferentes sensores são separados em coleções específicas, como os de pressão em uma e os de voltagem em outra.
    
    \item \textbf{Metadados}: Dedicado à armazenagem de metadados do sistema. Até o momento, a única coleção presente é a "AlertParameter", que reúne parâmetros utilizados para gerar alertas associados a cada sensor.
\end{itemize}

Com esta estruturação, busca-se não apenas organizar de forma lógica os dados, mas também otimizar operações de consulta e garantir uma expansão simplificada à medida que novas necessidades de armazenamento emergem no sistema. 

A implementação do acesso ao banco de dados está detalhada na seção de implementação da API, em ~\ref{subsubsec:DatabaseImpl}.


