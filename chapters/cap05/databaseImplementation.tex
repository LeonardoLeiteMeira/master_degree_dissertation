\section[Database Implementation]{Database Implementation}

Within the system implementation, MongoDB was used to store all system information. This non-relational database allowed for flexible data organization, facilitating the storage of different data that can be received by the data receiving module, and facilitating the creation of processing layers.

Regarding data organization, the following databases were created:

\begin{itemize}
    \item \textbf{Users}: Stores information related to users. It has collections that record login attempts, users' personal details, and tokens associated with them.
    
    \item \textbf{Notification}: Intended for system notifications. Currently, this database only contains notifications associated with machine alerts, generated by the data received from the sensors along with the stored parameters.
    
    \item \textbf{Raw Data}: This database is dedicated to storing raw data from different sensors. Each type of sensor, such as pressure sensors, has its own collection, ensuring a grouping of information that facilitates analysis. If a new type of sensor is created, the system will create a new collection automatically.
    
    \item \textbf{Processed Data}: As the name suggests, it stores data that has already undergone a processing stage. Thus, interpreted data from different sensors are separated into specific collections, such as pressure in one and voltage in another.
    
    \item \textbf{Metadata}: Dedicated to storing system metadata. So far, the only collection present is "AlertParameter", which gathers parameters used to generate alerts associated with each sensor.
    
    \item \textbf{Downtime}: Stores one collections with processed data. This database with this collection is only to simulate what the machine downtime data would look like if they were inserted into the system.
\end{itemize}

With this structuring, the aim is not only to logically organize the data, but also to optimize query operations and ensure simplified expansion as it becomes necessary to store new data in the system.

The implementation of database access is detailed in the implementation section of the \gls{API}, in ~\ref{subsubsec:DatabaseImpl}.