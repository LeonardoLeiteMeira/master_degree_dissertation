\section{Web Server Implementation}\label{sec:webserverimpl}
The construction of the Web Server was carried out using NGINX \cite{nginxDocs} as the main entry platform for all requests, serving both the user interface (front-end) and the API. Different endpoints are defined to direct appropriate requests to the corresponding services, whether the API or the user interface.

For requests that are different from HTTP methods, there are specific settings, whether for connection via stream, which is used for real-time data transmissions, as well as a setting for communications via web socket, to send notifications. The configuration for web socket has a specific set of proxy directives, including the upgrade of the connection to support web sockets.

The complete NGINX configuration can be found in the \texttt{.conf} file in \ref{apendice1nginx}.

To ensure that communications are secure and protected, NGINX uses Let's Encrypt \cite{letsencrypt2023}. Let's Encrypt is a free, automated, and open certificate authority, which provides SSL/TLS certificates to enable HTTPS connections. Evidence of this can be found in the configuration where the paths to the SSL certificates are provided, specifically in \texttt{ssl\_certificate} and \texttt{ssl\_certificate\_key}.


A script named "\texttt{init-letsencrypt.sh}" is used to automate the configuration of Let's Encrypt within a Docker environment, ensuring a simpler implementation. This script is also attached for analysis \ref{letsencryptscript}.

Therefore, the NGINX configuration is divided into three main blocks, each with a specific purpose:

\begin{itemize}
    \item The server starts by listening on port 80 (standard HTTP) and redirects all requests to the HTTPS version of the site, using port 443.
    \item The SSL configuration is defined in the second server block, listening on port 443 (standard HTTPS). The certificates provided by Let's Encrypt are referenced here.
    \item Different endpoints are defined to direct requests to the corresponding services, whether it's the API, real-time broadcasts, or the user interface.
\end{itemize}