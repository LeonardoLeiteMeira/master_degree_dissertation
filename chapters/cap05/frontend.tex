\section[Frontend Implementation]{Frontend Implementation}\label{sec:implFront}

The implementation of the user interface was developed according to the architecture exposed in section \ref{sec:archFront}. The development of the \textit{frontend} is segmented into several parts, which include the organization of the system pages according to the pre-defined structure of Next.js \cite{nextjsDocs}, the management of data accessed by the components, the configuration for external access, and the construction of individual components.

It is relevant to note that, to maintain compliance with best practices and simplify development, the default settings of Next.js were maintained.

\subsection{System pages}\label{subsec:}
Being a framework, NextJs has a predefined structure for creating system pages as well as their routes \cite{nextjsDefiningRoutes}. Within the framework files, the \textit{pages} folder is used to store each of the system's pages, with each file being a page, and the file name being the route for access. The configuration of the pages occurs through files with specific names, in this case \textit{\_app.tsx} and \textit{\_document.tsx}.

\subsubsection{Configuration pages}\label{subsec:configPage}
Regarding the configuration pages, we first have the \textit{\_app.tsx}. This file is responsible for configuring and managing contexts, global styling, and date localization, in other words, global aspects of the entire application.

The code begins with the import of various modules and libraries, which includes specific contexts such as \texttt{OpenContext} and \texttt{PrivateContext}, which are better explained in \ref{subsubsec:contextCreation}, and support for date localization with \texttt{AdapterDayjs} \cite{dayJsInstallation}.

The \texttt{LocalizationProvider} and \texttt{AdapterDayjs} are from libraries that aim to provide localization and date formatting functionalities. The \texttt{LocalizationProvider} acts as a wrapper for the date system, allowing integration with different date management libraries. In this case, the \texttt{AdapterDayjs} is used as the adapter for the Day.js library, allowing dates to be manipulated and formatted efficiently and compatible with various geographical locations and formats. With these libraries, it becomes easier to manage dates for the construction of the dashboard filter that manages the date period displayed in the charts, explained in \ref{sec:histicalGraphs}.

The type \texttt{NextPageWithLayout} was defined to enrich the page properties with information about the layout. This allows each page to have a custom layout if necessary, offering great flexibility in interface design.

The main function \texttt{App}, which receives \texttt{Component} and \texttt{pageProps} as arguments, is responsible for setting up the layout and rendering the page components. The logic within this function checks the current route using \texttt{useRouter} \cite{nextjsUseRouter} to determine if the user is on the login page.

The content is then encapsulated within the relevant contexts. If the user is on the login page, only the \texttt{OpenContext} is applied. For all other pages, the \texttt{PrivateContext} is additionally applied, ensuring that sensitive information is accessed only by authenticated users. The contexts used are detailed in \ref{subsec:contextApi}.

Within the \texttt{LocalizationProvider}, the \texttt{AdapterDayjs} adapter is used to provide date localization functionalities, making the application more versatile in different locations.


\begin{verbatim}
import {OpenContext, PrivateContext} from '@/context'
import '@/styles/globals.css'
import { LocalizationProvider } from '@mui/x-date-pickers'
import { AdapterDayjs } from '@mui/x-date-pickers/AdapterDayjs'
import { NextPage } from 'next'
import type { AppProps } from 'next/app'
import { useRouter } from 'next/router'
import { ReactElement, ReactNode } from 'react'

export type NextPageWithLayout<P = {}, IP = P> = NextPage<P, IP> & {
    getLayout?: (page: ReactElement) => ReactNode
}

type AppPropsWithLayout = AppProps & {
    Component: NextPageWithLayout
}

export default function App({ Component, pageProps }:
AppPropsWithLayout) {

    const getLayout = Component.getLayout || ((page) => page)
    const router = useRouter()
    const isLoginPage = router.pathname === "/"

    const componentWithProps = <Component {...pageProps} /> 

    return getLayout(
    <LocalizationProvider dateAdapter={AdapterDayjs}>
        <OpenContext>

        {isLoginPage?
            <>{componentWithProps}</>
            :<PrivateContext>
                {componentWithProps}
            </PrivateContext>
        }

        </OpenContext>  
    </LocalizationProvider>
    )
}
\end{verbatim}

Although it is a simpler file compared to \texttt{\_app.tsx}, the \texttt{\_document.tsx} has the responsibility of defining the global HTML structure of the application.

In the file, the components \texttt{Html}, \texttt{Head}, \texttt{Main}, and \texttt{NextScript} from the \texttt{next/document} library were imported. These components are used to create the basic structure of the HTML page within NextJs.

The \texttt{Html} component is used to encapsulate all HTML content and includes the attribute \texttt{lang="en"}, which sets the page language to English. The \texttt{Head} component \cite{nextjsHeadComponent} is employed to add elements to the HTML page header. In this case, the page title is set to 'Dashboard'.

The body of the HTML page is composed of the \texttt{Main} and \texttt{NextScript} components. The \texttt{Main} is where the main content of the page is inserted, while the \texttt{NextScript} is responsible for including the necessary scripts for Next.js to operate.

It is worth noting that the \texttt{\_document.tsx} does not have access to specific page features such as the \texttt{getInitialProps} \cite{nextjsInitialProps}, \texttt{getStaticProps} \cite{nextjsGetStaticProps}, or \texttt{getServerSideProps} \cite{nextjsGetServerSideProps} methods (NextJs functions for server-side data loading). This implies that this file is ideal for configurations that are common to all pages and do not require dynamic information.

\begin{verbatim}
import { Html, Head, Main, NextScript } from 'next/document'

export default function Document() {
  return (

    <Html lang="en">
      <Head title='Dashboard'/>
      <body>
        <Main />
        <NextScript />
      </body>
    </Html>
  )
}
\end{verbatim}

\subsubsection{System Pages}\label{subsec:}
The system pages are divided into two types, private and public, with the public one being only the login page. This public page is in the \texttt{index.tsx} file, being the root route of the system.

In this file, only \texttt{meta} settings and the \texttt{Login} component are invoked. The \texttt{Head} element \cite{nextjsHeadComponent} is used to define global HTML settings, such as the page title and metadata.

The \texttt{Login} component is called within the \texttt{main} tag, which serves as the main content of the page. This design approach keeps the \texttt{index.tsx} page lean, transferring most of the logic and visual presentation to the \texttt{Login} component. This is an example of the principle of separation of concerns, where each file or component has a single clearly defined responsibility.

\begin{verbatim}
export default function Home() {
    return (
      <>
        <Head>
          <title>Catraport Dashboard</title>
          <meta name="description" 
            content="Generated by create next app" />
          <meta name="viewport" 
            content="width=device-width,
            initial-scale=1" />
          <link rel="icon" href="/favicon.ico" />
        </Head>
        <main>
          <Login/>
        </main>
      </>
    )
  }
\end{verbatim}

The other system pages are located within the \texttt{dashboard} folder, which is also inside the \texttt{pages} folder. This implies that all pages within this folder should be accessed at the \texttt{/dashboard} route \cite{nextjsDefiningRoutes}.

Within the dashboard, there is the main page, at \texttt{/index.tsx}, with the dashboard page that displays real-time data and charts with processed historical data. The functionalities of this page are detailed in \ref{cap:functions}.

This file follows the same design logic observed on the login page, maintaining the separation between the page settings and the logic of the invoked components.

The \texttt{Dashboard} component is based on composition, delegating various responsibilities to individual components. The \texttt{DashboardLayout} component is used as a container that defines the global structure of the page, providing a consistent layout for the other dashboard pages as well. Within this component, several others are called to perform specific functions.

The \texttt{DashboardHeader} is responsible for displaying the page header, providing access to filters for viewing information. Following this is the \texttt{SensorsValues} component, which is designated to display the sensor values in real time.

A \texttt{Divider} element, from the \texttt{Material UI 5} library \cite{muiDocs}, is inserted to provide a visual separation between the different sections of the page. Finally, the \texttt{SensorsGraphs} component is invoked to display graphs related to the historical data of the sensors in an aggregated manner.

\begin{verbatim}
export default function Dashboard() {
    return (
        <DashboardLayout>
            <DashboardHeader/>
            <SensorsValues/>
            <Divider/>
            <SensorsGraphs/>
        </DashboardLayout>
    )
}
\end{verbatim}

The other pages of the dashboard were also built using the demonstrated composition logic and using the same base component for the layout, \texttt{DashboardLayout}. These pages are:
\begin{enumerate}
    \item \textbf{Maintenace}: Responsible for displaying machine downtime data in the form of graphs to demonstrate the visualization of this information within the system, \ref{sec:downtime}.
    \item \textbf{Profile}: Responsible for displaying the information of the user who is logged into the system, as well as allowing changes to the data, \ref{sec:profile}.
\end{enumerate}


\subsection{System data management}\label{subsec:contextApi}
One of the important points in frontend development is the management of global states, which are information or behaviors shared among unrelated components. Within this project, the \texttt{Context API} \cite{reactCreateContext} was used for this purpose, being a native React solution that stands out for its ease of implementation and use. This feature allows data to be efficiently passed throughout the component tree, eliminating the need to manually pass properties through intermediate levels.

Not only data states, but also behavioral states, such as the user's login status and the side menu status (open or closed), were managed through the \texttt{Context API}. The data layer was constructed in such a way that all necessary information for the system's operation, which depends on a global value, was included.

The contexts created for state management are as follows:
\begin{enumerate}
    \item \texttt{SnackbarContext}: Used for managing messages and alerts in the system, facilitating access to the function that displays alerts and its configuration throughout the system.
    \item \texttt{AuthContext}: Responsible for managing the user's authentication state, \ref{subsubsec:auth}.
    \item \texttt{ThemeContext}: Responsible for managing the visual theme of the application \cite{muiDefaultTheme}.
    \item \texttt{NotificationContext}: Used for the management, reading, and receiving of notifications in the system.
    \item \texttt{LegacyContext}: Manages machine stoppage information after being read from the backend.
    \item \texttt{IotContext}: Used for managing states related to IoT devices in the system.
    \item \texttt{DrawerContext}: Responsible for managing the state of the side menu (open or closed), and making it available on all system pages that use the side menu.
\end{enumerate}

Each context was designed with a specific function in order to allow a clear separation of responsibilities. Using the architecture shown in X, this frontend data management model becomes simple and scalable, as specified in the requirements, in \ref{sec:req}.

\subsubsection{Management of different contexts}\label{subsubsec:difContexts}
To deal with the complexity generated by the variety of contexts needed in the system, it was chosen to instantiate these contexts through specific components. These components were designed to encapsulate different groups of contexts, according to access needs.

Two main components were developed: \texttt{OpenContext} and \texttt{PrivateContext}. The former is responsible for instantiating contexts that are available to any individual who accesses the system. The latter is in charge of instantiating contexts that can only be accessed by authenticated users. The creation of a context is better exemplified in ~\ref{subsubsec:contextCreation}.

These components are used in the \texttt{\_\_app.tsx} file, which is the initial configuration file of Next.js, as explained in ~\ref{subsec:configPage}, in order to make the contexts accessible throughout the application's component tree.

Below is the code that illustrates how these components were implemented:

\begin{verbatim}
interface Props{
    children:React.ReactNode
}

function OpenContext({children}:Props){
    return (
        <SnackbarContextProvider>
            <AuthContextProvider>
                <ThemeContextProvider>
                        {children}
                </ThemeContextProvider>
            </AuthContextProvider>
        </SnackbarContextProvider>
    )
}

function PrivateContext({children}:Props){
    return(
        <>
            <NotificationProvider>
                <LegacyContext>
                    <IotContext>
                        <DrawerContextProvider>
                            {children}
                        </DrawerContextProvider>
                    </IotContext>
                </LegacyContext>
            </NotificationProvider>
        </> 
    )
}
\end{verbatim}

In this way, contexts are properly isolated and managed, ensuring that the correct data and functionalities are available to users, according to their level of access.


\subsubsection{Creating a context}\label{subsubsec:contextCreation}
For the creation of a context, the correct definition of types is first necessary, mandatory due to the use of \texttt{Typescript}. For each context, a properties interface (\texttt{Props}) and a default value (\texttt{DEFAULT\_VALUE}) are created.

To demonstrate the creation of a context within the project, the \texttt{DrawerContext} will be used as an example, which is responsible for managing the state of the application's drawer. The following code exemplifies how this context was created:

\begin{verbatim}
import { createContext, useContext, useState } from "react"

interface Props {
    open:boolean
    setOpen:React.Dispatch<React.SetStateAction<boolean>>
}

const DEFAULT_VALUE = {
    open:false,
    setOpen:()=>{}
}

const DrawerContext = createContext<Props>(DEFAULT_VALUE)

function DrawerContextProvider({ children }:{children:React.ReactNode}){
    const [open, setOpen] = useState<boolean>(false)

    return (
        <DrawerContext.Provider value={{open,setOpen}}>
            {children}
        </DrawerContext.Provider>
    )
}

export default function useDrawer(){
    return useContext(DrawerContext)
}

export {DrawerContextProvider};
\end{verbatim}

In this example, the \texttt{DrawerContext} context is created using the \texttt{createContext} function from React \cite{reactCreateContext}. The \texttt{Props} interface defines the types for the open state of the drawer (\texttt{open}) and the function to set this state (\texttt{setOpen}). A default value (\texttt{DEFAULT\_VALUE}) is established to initialize the context.

The \texttt{DrawerContextProvider} component uses local React state to manage the \texttt{open} state value. This value and the \texttt{setOpen} function are then made available to all child components through the \texttt{DrawerContext.Provider}. In this way, the \texttt{DrawerContextProvider} component is used in context management, as explained in ~\ref{subsubsec:difContexts}, to make all information accessible to the entire component tree.

The \texttt{useDrawer} function is a custom function that facilitates access to the \texttt{DrawerContext} in any part of the application. Within React, functions with 'use' in front are called \texttt{hooks}, as can be seen in the official documentation at \cite{reactHooksReference}.

Thus, the context was created and can be used to manage the open and closed state of the side menu throughout the application.

This context creation model is repeated for all other contexts, with the addition of access to the backend, which is explained in ~\ref{subsec:api_access}.

\subsection{External Access}\label{subsec:api_access}
For interaction with data stored and managed by the backend, an external access layer was established in the frontend. This layer serves as a centralized point for all network requests and is necessary for reading and manipulating data that is outside the scope of the frontend, therefore it deals with HTTP requests, WebSocket connections, and data reception via stream.

The various contexts created in the system, as described in ~\ref{subsubsec:contextCreation}, use this external access layer to load the necessary data. At the initialization of each context, function calls to this layer are made, if necessary. These calls are responsible for making requests to the backend and for receiving the returned information.

In the example below, the hook \texttt{useEffect} \cite{reactUseEffect} is used to call a function that accesses the external access layer to load data related to the sensors, being them the real-time data, and chart data, as soon as the context is initialized.

\begin{verbatim}
const fetchSensorData = useCallback(async()=>{
    await Promise.all([
        getRealTimeDataData(reciveRealTimeData),
        fetchGraphData(),
    ])
},[])

useEffect(()=>{
    fetchSensorData()
},[])
\end{verbatim}


\subsubsection{HTTP Requests}\label{subsubsec:httpRequest}
The \texttt{Axios} library \cite{axiosIntro} was used to facilitate the making of requests to the backend. This library provides a simple and efficient interface for creating \gls{HTTP} requests, and it was integrated into the functions of the external access layer. This library allows an initial configuration to be used in all the requests made.

In the configuration used, the backend address and the URL to add the settings to the base route, where all requests should go, are read from the system's environment variables. Another applied configuration is the addition of the interceptor \cite{axiosInterceptors} at the moment the request is made, adding the necessary configuration for authentication to the header, reading the access token from the local storage \cite{mdnLocalStorage}, and adding it in the correct format for backend reading in requests that require authentication.

\begin{verbatim}
const baseUrl = process.env.NEXT_PUBLIC_API_URL
const apiRoute = process.env.NEXT_PUBLIC_API_ROUTE

const baseApi = axios.create({
  baseURL: `http://${baseUrl}${apiRoute}`
});

baseApi.interceptors.request.use(function (config) {
    let token = localStorage.getItem("access_token")
    if (token) {
      config.headers['Authorization'] = `Bearer ${token}`;
    }
    return config;
  }, function (error) {
    return Promise.reject(error);
});
\end{verbatim}

The functions of this layer use this base configuration to fetch data from the backend. The function below exemplifies one of these external access functions to fetch data for a specific chart. The function uses the base configuration of \texttt{axios} along with a specific url to access the desired endpoint to fetch the information. If the return has a \texttt{status code} equal to 200, it means that the request was successful, then the data is returned to the context that made the call of this function. The context receives the data and makes it available to the entire application, as explained in ~\ref{subsec:contextApi}.

\begin{verbatim}
async function getGraphData(
  machine:string,
  type:SensorType,
  startDate:Dayjs,
  endDate:Dayjs):Promise<Array<MachineGraphAggregateData>>{
  try{
    let url = "/iot/graph_info"+get_graph_query(
      machine,
      type,
      startDate,
      endDate)
    let response = await baseApi.get(url)
    if(response.status === 200){
      return response.data
    }
    throw("Error to access graph - "+response.status)
  }catch(ex){
    throw("Error to access graph - "+ex)
  }
}
\end{verbatim}


\subsubsection{WebSocket Connection}\label{subsec:websocketConncetion}
In addition to \gls{HTTP} requests, the external access layer also manages the WebSocket connection. Specifically, the notification context uses this connection to receive and manage real-time notifications.

The management of the WebSocket connection is carried out through the \textit{CustomSocketConnection} class. This class is designed as a singleton, ensuring that a single instance is created and reused throughout the application. It is responsible for initializing and maintaining the \texttt{Socket} object, which is part of the \textit{socket.io-client} library \cite{socketIoClientApi}.

\begin{verbatim}
class CustomSocketConnection{
  private static _instance:CustomSocketConnection|null = null
  private _socketio: Socket|null = null

  private constructor() {
      if (CustomSocketConnection._instance === null) {
          CustomSocketConnection._instance = this;
      }
  }

  public static getInstance(): CustomSocketConnection {
      if (!CustomSocketConnection._instance) {
          CustomSocketConnection._instance = new CustomSocketConnection();
      }
      return CustomSocketConnection._instance;
  }

  get socketio(){
    if(this._socketio===null){
      let token = localStorage.getItem("access_token")
      let url = process.env.NEXT_PUBLIC_API_URL??"localhost"
      this._socketio = io(url, { autoConnect: false, auth: { "Authorization": token } })
    }
    return this._socketio
  }
}
\end{verbatim}

The \texttt{getInstance()} method ensures that only one instance of the class is created. This instance is stored as a static attribute and is returned whenever requested.

An instance of the \textit{CustomSocketConnection} class is used in the notification context. Through this instance, the context is able to receive messages from the server, manipulate them, and then make them available to the entire application.

The class also includes an authentication mechanism. The access token is retrieved from local storage \cite{mdnLocalStorage} and is used as part of the authentication header during the connection process.

Within the notification context, the Socket attribute of the class is used in the context initialization to establish the connection. The instance of the class is stored in a constant, and then three functions are registered for connection events, notification receipt, and disconnection. The function that handles the event of receiving a new notification sends the data to a helper function whose purpose is to analyze the received data and update the list of notifications that appear on the screen.

\begin{verbatim}
useEffect(() => {
  const socket = CustomSocketConnection.getInstance()
  function onConnect () {
      console.log(`Connected with id: ${socket.socketio.id}`);
      console.log(`Connection Status: ${socket.socketio.connected}`);
  }

  function newNotification(data:any){
      console.log("Socket Io Notification", data);
      checkNewNotification(data);
  }

  function disconnect(data:any) {
      console.log("Socket Io disconnect", data);
      console.log("Connection Status");
  }

  socket.socketio.on('connect', onConnect);
  socket.socketio.on('Notification', newNotification);
  socket.socketio.on('disconnect', disconnect);

  return () => {
      socket.socketio.off('connect', onConnect);
      socket.socketio.off('Notification', newNotification);
      socket.socketio.off('disconnect', disconnect);
  };
}, [notifications]);
\end{verbatim}

In this way, the external access layer provides a Web Socket connection method to be used to receive new notifications while the user is connected to the system.

\subsubsection{Receiving data via Stream}\label{subsec:streamData}

Regarding the external access layer, the last type of connection is the receipt of data via stream. This mechanism allows for the constant updating of received data, thus ensuring a continuous flow of updated information for the application. Specifically, this connection is used to provide the data from the sensors that are received in the backend, as explained in \ref{sec:Implementation of the data reception module}.

The receipt of data via stream is implemented using the \textit{fetch} function \cite{mdnFetchAPI}, native to JavaScript. This function is responsible for making the request to the corresponding endpoint and obtaining the data stream in real time.

The \texttt{readStream} function is used to interpret the data from the stream. This function receives the reader of the response body, and returns the data converted to the \gls{json} format.

\begin{verbatim}
async function readStream(
  reader:ReadableStreamDefaultReader<Uint8Array> | undefined) {
  let result = await reader?.read()
  if (!result?.done) {
      let value = result?.value
      if(value){
          const jsonData = parseBytesToJson(value)        
          return jsonData
      }
  }
  return false

}
\end{verbatim}

The function \texttt{getRealTimeDataData} is responsible for initiating the data streaming process. It is in this function that the IoT sensor data context makes the call and, consequently, receives and sends the data to the function received as a parameter.

Since the function does not use the axios structure explained in ~\ref{subsubsec:httpRequest}, it is necessary to perform the same authentication configuration that was previously explained in the structure.

After making the request, the data reader is read from the body of the request response, and passed to the \texttt{readStream} function, explained earlier, to be interpreted and converted to \gls{json}.

\begin{verbatim}
async function getRealTimeDataData(setData: UpdateDataFromStream) {
  try{
    let token = localStorage.getItem("access_token")
    let url = baseApi.getUri()
    let response = await fetch(`${url}/iot/realtime`,{headers:{
        Authorization: `Bearer ${token}`,
        'Content-Type': 'application/json',
    },});

    const reader = response?.body?.getReader();
    let result:Array<MachineRealTimeData>|boolean = await readStream(reader)
    if(typeof(result) === "boolean"){
        throw "Error to read stream data"
    }
    do{
      if (Array.isArray(result)) {
          const typedResult = result as MachineRealTimeData[];
          setData(typedResult);
      }
      result = await readStream(reader)
    }while(result!==false)
  }catch(ex){
    console.error(ex)
    throw ex
  }
}
\end{verbatim}

The function \texttt{getRealTimeDataData} continues to run as long as the connection is open and no errors occur. The \texttt{setData} method is then called to send the reading result to the context that called the function.

Within the context of IOT sensor data, the \texttt{setData} function passed as a parameter only updates the global state to update the information in all components that make use of it.

\subsection{Component construction}\label{subsec:componentization}
In the system implementation, the logic of component construction \cite{reactFirstComponent} was used to compose the screens. Modularity and code reuse are critical factors that motivate this choice.

Components access data directly from contexts, as I discuss in \ref{subsec:contextApi}. This method facilitates the passage of data and allows the components to be more specific in their function, in addition to avoiding the passage of many properties in the component tree.

The basis for the smaller components was extracted from the \gls{MUI5} library \cite{muiDocs}. This includes elements such as buttons, containers, text boxes, and others. A practical example is the sidebar menu on the dashboard pages, where the Drawer component from \gls{MUI5} was employed.

A component that deserves highlight is the \texttt{Grid} from \gls{MUI5} \cite{muiReactGrid}. The use of Grid allowed a simple spatial organization of the user interface elements. This grid system offers a flexible approach to allocate space, align content, and handle screen variations, which is especially useful in web applications with a lot of information and different components on screen. In the sidebar menu, for example, the \texttt{Grid} component was used to organize the menu information and define the positioning according to the open or closed state.

\begin{verbatim}
<Grid 
  container
  direction="column"
  justifyContent="space-between"
  alignItems={open?"center":"start"}
  height={"97vh"}
>
  <Grid 
    container
    item
    direction="column"
    justifyContent="space-between"
    alignItems={open?"center":"start"}
  >
    // Content
  </Grid>
</Grid>
\end{verbatim}

The use of Grid, therefore, contributed to the cohesion of the layout, providing a solid structure upon which other components could be organized in a simple and easy-to-understand manner, fulfilling requirements to facilitate maintenance specified in \ref{ssubec:reqNfuctional}.

On the other hand, for the construction of the graphs, the Recharts library was used. The graph shown on the dashboard with the information generated by the processing module, detailed REF \ref{sec:ImplModuloProcessamento}, is composed of Scatter, Area, Bar, and Line charts, thus allowing a multifaceted analysis of the generated information.

For components that require date manipulation, the \texttt{Days Js} library \cite{dayJsInstallation} was integrated. A use case is the date filter component for displaying graphs, which also employs the DatePicker \cite{muiDatePickerValidation} from \gls{MUI5} for a more intuitive user interface. The Days Js library facilitates the manipulation of date data input and output.

\begin{verbatim}
<DatePicker
  value={dateFilter.startDate}
  onChange={(value)=>onChangeDate(value, "startDate")}
  label="Data inicial"
/>
\end{verbatim}

\begin{verbatim}
const onChangeDate = (newDate:Dayjs|null, 
  dateField:"startDate"|"endDate")=>{
  if(newDate!==null){
    setIsDataUpdated(false)
    if(dateField==="endDate"){
      setDateFilter(oldValue=>({...oldValue,"endDate": newDate}))
    }else{
      setDateFilter(oldValue=>({...oldValue,"startDate": newDate}))
    }
  }
}
\end{verbatim}